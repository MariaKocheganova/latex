\documentclass[10pt]{beamer}
\usepackage{fancyvrb}
\setbeamertemplate{caption}[numbered]


\usepackage{color}
\usepackage[T2A]{fontenc}
\usepackage[russian]{babel}
\usepackage{amsmath}
\newcommand{\No}{\textnumero}

\usepackage{beton}

\usepackage{tikz}
\usepackage[utf8]{inputenc}

\defbeamertemplate{footline}{centered page number}
{%
  \hspace*{\fill}%
  \usebeamercolor[fg]{page number in head/foot}%
  \usebeamerfont{page number in head/foot}%
  \insertpagenumber\,/\,\insertpresentationendpage%
  \hspace*{\fill}\vskip2pt%
}
\setbeamertemplate{footline}[centered page number]
%\mode<presentation>{
%\usetheme{Rochester}
%}
\newcommand{\backupbegin}{
   \newcounter{framenumberappendix}
   \setcounter{framenumberappendix}{\value{framenumber}}
}
\newcommand{\backupend}{
   \addtocounter{framenumberappendix}{-\value{framenumber}}
   \addtocounter{framenumber}{\value{framenumberappendix}} 
}
\mode<presentation>{
%  \usetheme{Madrid}
  \usetheme{AnnArbor}
  \usefonttheme{serif}
}
\makeatletter
\setbeamertemplate{footline}
{
  \leavevmode%
  \hbox{%
  \begin{beamercolorbox}[wd=.333333\paperwidth,ht=2.25ex,dp=1ex,center]{author in head/foot}%
    \usebeamerfont{author in head/foot}\insertshortauthor%~~\beamer@ifempty{\insertshortinstitute}{}{(\insertshortinstitute)}
  \end{beamercolorbox}%
  \begin{beamercolorbox}[wd=.333333\paperwidth,ht=2.25ex,dp=1ex,center]{title in head/foot}%
    \usebeamerfont{title in head/foot}\insertshorttitle
  \end{beamercolorbox}%
  \begin{beamercolorbox}[wd=.333333\paperwidth,ht=2.25ex,dp=1ex,right]{date in head/foot}%
    \usebeamerfont{date in head/foot}\insertshortdate{}\hspace*{2em}
    \insertframenumber{} / \inserttotalframenumber\hspace*{2ex} 
  \end{beamercolorbox}}%
  \vskip0pt%
}
\makeatother


\begin{document}
% ывапывап
% \title[ЧИСЛЕННЫЙ АНАЛИЗ МАРКОВСКИХ...]{\normalsize \color{blue} ЧИСЛЕННЫЙ АНАЛИЗ МАРКОВСКИХ ПРОЦЕССОВ ГИБЕЛИ И \\     РАЗМНОЖЕНИЯ С ПОМОЩЬЮ ПРЕОБРАЗОВАНИЯ ЛАПЛАСА}
% \subtitle{Учебно-методическая разработка}


\author[В.М.~Кочеганов]{
\flushright Аспирант кафедры ПрИнж ИИТММ \\ \textbf{Кочеганов~В.М.}\\
\medskip
Направление: 01.06.01\\
Направленность: 01.01.05}
\institute[ГИА]{\normalsize Государственный экзамен}

\date{29.06.2018}
\begin{frame}
\titlepage
\end{frame}



\begin{frame}{Цели и задачи разработки}
\small{
\textbf{Целью} изучения пособия является знакомство с методами теории массового обслуживания студентов, не имевших спецкурса по теории массового обслуживания в бакалавриате.

Пособие \textbf{предназначено} для студентов магистратуры, обучающихся по направлению \textbf{«Прикладная математика и информатика»} и \textbf{может быть использовано} при чтении курса по выбору \textbf{«Аналитические и численные методы в теории очередей»}

\begin{block}{Формируемые компетенции:}
\begin{itemize}
    \item  ОК-1, способность к абстрактному мышлению;
\item ОПК-4, способность использовать и применять углубленные знания в области прикладной
математики;
\item ПК-2, способность разрабатывать и анализировать концептуальные и теоретические
модели решаемых научных проблем и задач;
\item ПК-3, способность разрабатывать и применять математические методы, системное и
прикладное программное обеспечение для решения задач научной и проектно-технологической
деятельности
\end{itemize}
\end{block}}
\end{frame}

\begin{frame}{Структура учебно-методической разработки}
Учебно-методическое пособие имеет следующую структуру:
\medskip
\begin{enumerate}
    \item Введение
    \medskip
    \item Марковские процессы гибели и
размножения с непрерывным временем
\setbeamertemplate{enumerate items}[default]
  \begin{enumerate}
    \item Период занятости. Преобразование Лапласа
    \item Задачи
  \end{enumerate}
    \end{enumerate}
\end{frame}

\section{Первая часть. Введение}
\begin{frame}{Первая часть. Введение}
Введение обладает важной целью: \textbf{актуализировать знания студента}, заинтересовать его. 
  \begin{figure}[h]
    \centering
    \pgfimage[height=5cm]{Cafe}
    \caption{Прикладная задача: обслуживание в кафе}
    \label{VK:fig:2}
  \end{figure}
\end{frame}

\begin{frame}{Знакомство с базовыми понятиями}
На этой <<игровой>> задаче студент знакомится с:
\begin{itemize}
    \item понятием \textbf{входного потока}, в частности, \textbf{ординарного потока без последействия};
    \item понятием \textbf{интенсивности входного потока};
    \item понятием \textbf{очереди};
    \item понятиями \textbf{обслуживания} требования, \textbf{интенсивности обслуживания};
    \item \textbf{дифференциальными уравнениями Колмогорова}, задающих закон распределения количества свободных приборов;
    \item элементами \textbf{имитационного моделирования}.
\end{itemize}
\end{frame}



\begin{frame}[fragile]
\frametitle{Сопровождение программным кодом}
В тексте пособия приведены листинги программ на Octave для наглядности излагаемого материала. 
\begin{columns}[T] % align columns
\begin{column}{0.55\textwidth}
\begin{Verbatim}[label=Решение уравнений Колмогорова,fontsize=\scriptsize,baselinestretch=1,frame=single,xleftmargin=1em]
function pdot = kolm(p,t)
  m=2;
  if (t<6) l=6+2/3*t; else l=14-2/3*t; endif;
  sz = length(p);
  A = zeros(sz,sz);
  A(1:sz+1:sz^2) = -l-m*(0:sz-1);
  A(2:sz+1:sz^2) = l;
  A(sz+1:sz+1:sz^2) = m*(1:sz-1);
  A(sz^2) = -(sz-1)*m;
  pdot = A*p;
endfunction;

N=10;
tt = 0:0.1:9;
pp0 = [1, zeros(1,N)];
pp = lsode("kolm", pp0, tt);
plot(tt,pp); ## распределение вероятностей
plot(tt,pp*(0:N)'); ## среднее число клиентов
\end{Verbatim}
\end{column}
\hfill%
\begin{column}{0.45\textwidth}
\begin{figure}[h]
    \centering
    \pgfimage[height=4.3 cm]{distribution_tables_1}
    \caption{Динамика распределения числа свободных столиков}
  \end{figure}
\end{column}
\end{columns}
\end{frame}

\section{Вторая часть. Марковские процессы гибели и ...}
\begin{frame}[fragile]
\frametitle{Вторая часть. Марковские процессы гибели и размножения с непрерывным временем}
Более подробно разбирается марковский процесс гибели и размножения:
\begin{figure}[h]
    \centering
    \pgfimage[height=3cm]{Birth_death}
    \caption{Схема переходов для процесса гибели и размножения}
    \label{VK:fig:2}
  \end{figure}
\end{frame}


\begin{frame}[fragile]
\frametitle{Базовые определения}
\begin{block}{1. Связь с <<игровой>> задачей}
\begin{itemize}
    \item 
Проводится параллель с уже рассмотренной во введении задачей о кафе. Аналогичными рассуждениями выписываются формулы для допределного и предельного распределений количества свободных обслуживающих устройств.
\end{itemize}
\end{block}

\medskip
\medskip

\begin{block}{2. Знакомство с новыми понятиями}
\begin{itemize}
    \item \textbf{период занятости} системы;
    \item \textbf{преобразование Лапласа} для производящей функции распределения количества свободных обслуживающих устройств;
    \item \textbf{преобразование Лапласа} неотрицательной случайной величины.
\end{itemize}
\end{block}
\end{frame}

\begin{frame}[fragile]
\frametitle{Период занятости системы}
\begin{block}{3. Анализ системы M/M/1/$\infty$}
Основная часть главы посвящена поиску распределения \textit{периода занятости} системы двумя разными способами:
\begin{itemize}
    \item рассматривая систему как марковский процесс с непрерывным временем;
    \item рассматривая систему только в моменты скачков (прибытия или ухода требований), используя понятие \textbf{марковской цепи}.
\end{itemize}
\end{block}
\end{frame}

\begin{frame}[fragile]
\frametitle{Наглядность изложения}


%Получая на выходе распределение в виде преобразования Лапласа, приводится численный метод для его обращения. Численный метод сопровождается соответствующей программой на языке Octave:
Изложение сопровождается листингами из программ на языке Octave.
\medskip
\begin{columns}[T]
\begin{column}{0.5\textwidth}
\begin{Verbatim}[fontsize=\scriptsize,label=Обращение преобразования Лапласа,frame=single]
function ans = talbot(F, t,  N)
  r = 2*N./(5*mean(t));
  h = pi / N;
  k=1:(N-1);
  [xt, xk] = meshgrid(t,k);
  thk = xk*pi/N;
  s = r*thk.*(i+cot(thk));
  sgm = thk + (thk.*cot(thk)-1).*cot(thk);
  ans = r/N*( feval(F,r)*exp(r*t)/2+ ...
        sum( real( (feval(F,s).* (1+ ...
        i*sgm)).*exp(xt.*s)), 1 ) );
endfunction
\end{Verbatim}
\end{column}
\begin{column}{0.5\textwidth}
\begin{Verbatim}[fontsize=\scriptsize,label=Конкретное распределение,frame=single]
function y = F(s)
  mu = 0.6;
  lmb = 0.3;
  y = 2*mu./(lmb+mu+s+sqrt((sqrt(lmb)-...
       sqrt(mu))^2+s) .*sqrt((sqrt(lmb)+ ...
       sqrt(mu))^2+s));
endfunction

mu = 0.6;
lmb = 0.3;
tt = 0.01:0.2:5;
xtrue = sqrt(mu/lmb)*exp(-(lmb+mu)*tt) ...
        .*besseli(1,2*sqrt(lmb*mu)*tt)./tt;
xx = talbot(@F,tt,16);
plot(tt,xtrue,'-',tt,xx,'+')
\end{Verbatim}
\end{column}
\end{columns}
\end{frame}

\begin{frame}[fragile]
\frametitle{Вторая часть. Задачи}

В заключение пособия, студенту предлагается решить несколько задач на закрепление материала. Задачи можно объединить в следующие две группы:
\begin{itemize}
    \item задачи на нахождения преобразования Лапласа для известных распределений;
    \item задачи на проведение анализа для схожих систем массового обслуживания (численное решение дифференциальных уравнений Колмогорова, поиск стационарного распределения и распределения периода занятости и т.д.) .
\end{itemize}

\end{frame}

\end{document}

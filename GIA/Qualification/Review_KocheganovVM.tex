\documentclass[a4paper,14pt]{extarticle}
\usepackage[utf8]{inputenc}
\usepackage[T1]{fontenc}
\usepackage[russian]{babel}
\usepackage[onehalfspacing]{setspace}
\usepackage[margin=2cm]{geometry}
%\usepackage{pscyr}
\begin{document}
%\thispagestyle{empty}
\begin{center}
  \textbf{РЕЦЕНЗИЯ} \medskip 
  \\ на научно-квалификационную работу аспиранта кафедры\\ программной инженерии
  \textbf{Кочеганова Виктора Михайловича}\\ на тему <<Моделирование и анализ тандема систем массового обслуживания с циклическим управлением с продлением
>>
\end{center}

В своей научно-квалификационной работе Кочеганов~В.~М. занимался
исследованием управляющей системой массового обслуживания, являющейся
адекватной моделью различных реальных процессов управления транспортом
на перекрестках, вычислительных процессов, процессов передачи и
обработки информации. Существенной особенностью таких систем является
случайность в поступлении порций работы для обслуживающего
устройства. Поэтому основными используемыми разделами математики стали
теория вероятностей, исследование операций,  математическая статистика и теория случайных процессов.

Рецензируемая работа состоит из четырех глав. В первой главе формализуются свойства системы в терминах управляющих кибернетических систем и систем массового обслуживания. На основании этого строится математическая модель в виде вероятностного пространства с определенными на нем случайными величинами и элементами. Доказывается ряд следствий о виде некотороых условных распределений.

Вторая глава содержит анализ общей пятимерной марковской цепи. Доказывается свойство марковости и проводится классификация состояний этой марковской цепи. Для выделения существенных состояний доказывается ряд лемм о сообщающихся состояниях системы. Результатом главы является теорема с конкретным видом всех существенных состояний пятимерной марковской цепи.

В третьей главе исследуются случайные последовательности, полученные из общей путем удаления двух и трех компонент. Таким образом, получены результаты для последовательности, содержащей только состояние прибора и промежуточной очереди. При помощи итеративно-мажорантного метода найдены условия существования стационарного распределения марковской цепи, содержащей состояние прибора и состояние низкоприоритетной очереди. Аналогичные результаты получены для марковской цепи, содержащей состояние прибора и состояния очередей первичных требований.

Четвертая глава посвящена имитационному моделированию системы. Здесь приведено описание имитационной модели и описан алгоритм определения стационарного режима. Подробно представлены серии экспериментов, которые были проведены. Выводы, полученные из анализа системы, подтверждают теоретические изыскания, проделанные в остальных частях научно-квалификационной работы.

Все выводы работы строго обоснованы. Работа выполнена на высоком
математическом уровне. В качестве замечаний отмечу отсутствие ссылок в тексте на некоторые работы из списка литературы, а также употребление термина <<управляющая система>> вместо термина <<система массового обслуживания>>, что затрудняет понимание работы.

Считаю, что работа Кочеганова Виктора Михайловича удовлетворяет всем требованиям научно-квалификационных работ.  

\bigskip
\bigskip
\noindent

Рецензент



\end{document}


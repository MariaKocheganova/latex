\documentclass{report}
\usepackage[utf8]{inputenc}
\usepackage[russian]{babel}
\usepackage{amsmath} % for pmatrix
\usepackage{amssymb} % for \geqslant
\usepackage{geometry}
\usepackage{pgffor}
 \geometry{
 a4paper,
 total={170mm,257mm},
 left=10mm,
 right=10mm,
 top=3mm,
 bottom=1mm
 }
 
\begin{document}

 \foreach \n in {1,...,7}{
    \textbf{ Задание $1.$ } Приведите псевдокод для генерации дискретной случайной величины $\eta$, принимающей значения $-1$, $100$, $3000000$, $0.3$ с вероятностями $\frac{1}{3}$, $\frac{1}{5}$, $\frac{6}{15}$, $\frac{1}{15}$ соответственно. Для генерации следует использовать линейный конгруэнтный метод.
    
    \textbf{ Задание $2.$ }
    На заводе поломка станков происходит с интенсивностью $6$ станков в час (простейший пуассоновский поток). Время починки станка распределено по экспоненциальному закону с интенсивностью $8$ станков в час. Найдите стационарные вероятности марковского процесса $\{X(t, \omega): t\geqslant 0\}$, где $X(t, \cdot)$~--- случайная величина, характеризующая количество станков в очереди на починку в момент времени $t$.

 \textbf{ Задание $3.$ }
Постройте математическую модель $(\Omega, {\cal F}, {\mathbf P}(\cdot))$, которая бы описывала условие задачи~$2$.

    \noindent\makebox[\linewidth]{\rule{\paperwidth}{0.4pt}}
    
    }
     \pagebreak
  \foreach \n in {1,...,8}{  
  
        \textbf{ Задание $1.$ } 
    Сформулируйте линейный конгруэнтный метод. Для чего он нужен?
    
    \textbf{ Задание $2.$ }
    На звонки в call-центре отвечает Аннушка со скоростью $5$ звонков в минуту. В случае, если в очереди звонков накапливается больше $5$ человек, ей начинает помогать ее начальник Нестор Петрович с интенсивностью $3$ звонка в минуту. Процесс поступления звонков образует простейший пуассоновский поток с интенсивностью $7$ звонков в минуту. Постройте уравнения баланса для нахождения стационарных вероятностей для процесса изменения длины очереди клиентов.

 \textbf{ Задание $3.$ }
Решите уравнения баланса, полученные в задаче 2.
 
    \noindent\makebox[\linewidth]{\rule{\paperwidth}{0.4pt}}
    
  }
    
\end{document}

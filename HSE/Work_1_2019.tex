\documentclass{report}
\usepackage[utf8]{inputenc}
\usepackage[russian]{babel}
\usepackage{amsmath} % for pmatrix
\usepackage{amssymb} % for \geqslant
\usepackage{geometry}
\usepackage{pgffor}
 \geometry{
 a4paper,
 total={170mm,257mm},
 left=10mm,
 right=10mm,
 top=3mm,
 bottom=1mm
 }
 
\begin{document}

 \foreach \n in {1,...,6}{
    \textbf{ Задание $1.$ } Монета подбрасывается много раз. Вероятность успеха при каждом подбрасывании $p=0.8$. Построить математическую модель, в рамках которой исследователь сможет ответить на вопрос: какова вероятность, что через $n$ подбрасываний количество успехов будет делиться на 3. Посчитать эту вероятность для $n=4$.
    
    \textbf{ Задание $2.$ }
    Пусть на вероятностном пространстве $(\Omega, {\cal F}, {\mathbf P}(\cdot))$ задана марковская цепь $\{X_i(\omega);  i \geqslant 0\}$ с матрицей переходных вероятностей ${\mathbb{P}} $:
    $$
    {\mathbb{P}} = 
\begin{pmatrix} 
0.7 & 0.2 & 0.1 \\
0.2 & 0.6 & 0.2 \\
0.1 & 0.4 & 0.5 
\end{pmatrix}
$$
Обосновать, почему марковская цепь $\{X_i(\omega);  i \geqslant 0\}$ является эргодической. Найти ее эргодическое распределение.

 \textbf{ Задание $3.$ }
 Сформулируйте определение сообщающихся состояний марковской цепи.
 
    \noindent\makebox[\linewidth]{\rule{\paperwidth}{0.4pt}}
    
    }
  \foreach \n in {1,...,5}{  
        \textbf{ Задание $1.$ } Игральная кость подбрасывается большое число раз. Построить математическую модель, в рамках которой исследователь сможет ответить на вопрос: какова вероятность, что через $n$ подбрасываний накопленная сумма выпавших очков будет делиться на $4$.  Посчитать эту вероятность для $n=2$ (в общем случае находить эту вероятность не нужно).
    
    \textbf{ Задание $2.$ }
    Рассматривается случайное блуждание с нижней границей $0$ и верхней границей $4$: частица в каждый момент времени может находиться в одном из пяти состояний: $\{0, 1, 2, 3, 4\}$~--- и с некоторой вероятностью $p=0.7$ переходить в состояние с бОльшим номером или с вероятностью $q=1-p=0.3$~--- с мЕньшим. Достигнув нижней или верхней границы частица отражается. Предположим, что для этого эксперимента уже построено вероятностное пространство $(\Omega, {\cal F}, {\mathbf P}(\cdot))$ и на нем задана марковская цепь $\{X_i(\omega);  i \geqslant 0\}$, где $X_i(\omega)$~--- номер состояния частицы.
    
    Для марковской цепи $\{X_i(\omega);  i \geqslant 0\}$ необходимо построить матрицу переходных вероятностей, провести классификацию ее состояний (найти классы сообщающихся состояний, определить их периодичность, возвратность/невозвратность). Определить, является ли марковская цепь эргодической и, если да, найти ее эргодическое распределение.

 \textbf{ Задание $3.$ }
 Сформулировать и доказать уравнения Колмогорова--Чепмена для конечной марковской цепи.
 
    \noindent\makebox[\linewidth]{\rule{\paperwidth}{0.4pt}}
  }
    
\end{document}

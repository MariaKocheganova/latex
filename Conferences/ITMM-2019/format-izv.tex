%\usepackage[cp1251]{inputenc}
\usepackage{amsthm}
\usepackage{amsmath}
\usepackage{amssymb}
\usepackage{amsfonts}
\usepackage{mathtext}
\usepackage{mathrsfs}
\usepackage{cite}
\ifx\pdfoutput\undefined
\usepackage{graphicx}
\else
\usepackage[pdftex]{graphicx}
\fi

\usepackage{xypic}
\usepackage{epic}
%\usepackage{urwcyr}
%\usepackage{bng}
%\usepackage{pscyr}
%\usepackage[english,russian]{babel}
\usepackage[russian,english]{babel}
\usepackage[T2A]{fontenc}
\usepackage[utf8x]{inputenc}

\usepackage{multicol}
\premulticols=5pt \postmulticols=0mm \multicolsep=0mm
\usepackage{longtable}
%
%
%\usepackage{array}
%\usepackage{multirow}
%\renewcommand\multirowsetup{\centering}
%\usepackage{dcolumn}
\usepackage{graphicx}

% LAYOUT ----------------------------------------------------------
\usepackage{geometry}% Меняем поля страницы
\geometry{left=3.5cm}% левое поле
\geometry{right=1.5cm}% правое поле
\geometry{top=3cm}% верхнее поле
\geometry{bottom=2cm}% нижнее поле
\headsep=5mm%расстояние от верхнего колонтитула до текста
%
%Колонтитулы-----------------------------------------------------------
\usepackage{fancyhdr}%загрузим пакет
\pagestyle{fancy}%применим колонтитул
\fancyfoot{}\fancyhead{}% очистим футер (снизу) и хидер (сверху) на всякий случай
%\fancyhead[CE,CO]{\thepage}%номер страницы снизу по центру
\fancyfoot[LE,RO]{\thepage}%E - нечетные,  O - четные,  L - слева, R - справа, C - по центру
%\fancyhead[CO]{}%{текст-центр-нечетные}
%\fancyhead[LO]{Левый колонтитул}%
%\fancyhead[RE]{Правый колонтитул}%

%------------------------------------------------------------------

\makeatletter
\renewcommand{\@biblabel}[1]{#1.}

\renewcommand{\thesection}{\arabic{section}.}

\renewcommand{\section}{\@startsection
                                {section} %name
                                {1}       %level
                                {\z@}     %indent
                                {12pt}    %before skip
                                {10pt}    %after skip
                                {\reset@font\normalfont\bfseries\sffamily}}
                                          %font style

\newcommand{\titler}[1]{%
\vspace*{5mm}%

\noindent%
{\large\bf #1}%
\vspace*{5mm}%
}

\makeatother
\allowdisplaybreaks     % Разрешение переносить на другую страницу часть
                        % многострочной формулы
\sloppy                 %

\newcommand{\di}{\displaystyle}

%% аналог \- для внутритекстовых формул
%% пример: $y(x) \hm= R_\lambda f(x)
%% или : $Y(x)  \hm{:=} y_1(x) + y_2(x)
\newcommand{\hm}[1]{#1\nobreak\discretionary{}{\hbox{\ensuremath{#1}}}{}}
\relpenalty=10000 \binoppenalty=10000

%Обнулить все счетчики
\newcommand{\zerosetcounter}{
\setcounter{footnote}{0}\setcounter{section}{0}\setcounter{equation}{0}%
\setcounter{theorem}{0}\setcounter{lemma}{0}\setcounter{corollary}{0}\setcounter{prop}{0}
\setcounter{propos}{0}\setcounter{problem}{0}\setcounter{Theorem}{0}%
\setcounter{theoremnn}{0}\setcounter{lemmann}{0}\setcounter{corollarynn}{0}
\setcounter{propnn}{0}\setcounter{proposnn}{0}%
\setcounter{definitionn}{0}\setcounter{remarknn}{0}\setcounter{examplenn}{0}%
\setcounter{definition}{0}\setcounter{remark}{0}\setcounter{example}{0}%
\setcounter{hypothesis}{0}\setcounter{hypothesisnn}{0}\setcounter{theoremb}{0}
}

%%%%%%%%%%%%%%%%%%%%%%%%%%%%%%%%%%%%%%%%%%%%%%%%%%
%УДК
%%%%%%%%%%%%%%%%%%%%%%%%%%%%%%%%%%%%%%%%%%%%%%%%%%
\newcommand{\UDC}[1]{%
\vspace*{5mm}
\noindent\textsf{УДК %
#1}%
}

%%%%%%%%%%%%%%%%%%%%%%%%%%%%%%%%%%%%%%%%%%%%%%%%%%
%Заголовки статей
%%%%%%%%%%%%%%%%%%%%%%%%%%%%%%%%%%%%%%%%%%%%%%%%%%
\newcommand{\Rtitle}[1]{%
\begin{center}%
{ \bf\fontsize{14pt}{16pt}\sffamily%
#1}%
\end{center}%
}

\newcommand{\Etitle}[1]{%
\newpage
\begin{center}%
{ \bf\fontsize{12pt}{14pt}\sffamily%
#1}%
\end{center}%
}

%%%%%%%%%%%%%%%%%%%%%%%%%%%%%%%%%%%%%%%%%%%%%%%%%%%%%%%
%Авторы
%%%%%%%%%%%%%%%%%%%%%%%%%%%%%%%%%%%%%%%%%%%%%%%%%%%%%%%
\newcommand{\Rauthor}[1]{%
\centerline{%
\bf\fontsize{11pt}{14pt}
\sffamily%
#1}%
}

\newcommand{\Eauthor}[1]{%
\centerline{%
\bf\fontsize{11pt}{14pt}
\sffamily%
#1}%
}

%%%%%%%%%%%%%%%%%%%%%%%%%%%%%%%%%%%%%%%%%%%%%%%%%%%%%%%
%Сведения об авторах %affiliation
%%%%%%%%%%%%%%%%%%%%%%%%%%%%%%%%%%%%%%%%%%%%%%%%%%%%%%%
\newcommand{\Raffil}[1]{%
\begin{center}
\begin{minipage}{150mm}{%
\small\sffamily%
#1}%
\end{minipage}%
\end{center}
}

\newcommand{\Eaffil}[1]{%
\begin{center}%
\begin{minipage}{150mm}{%
\small\sffamily%
#1}%
\end{minipage}%
\end{center}
}

%%%%%%%%%%%%%%%%%%%%%%%%%%%%%%%%%%%%%%%%%%%%%%%%%%%%%%%
%E-mail
%%%%%%%%%%%%%%%%%%%%%%%%%%%%%%%%%%%%%%%%%%%%%%%%%%%%%%%
\newcommand{\Email}[1]{%
\vspace*{-3mm}
\centerline{%
\small\textit{%
E-mail: #1}}%
\vspace*{3mm}%
}

%%%%%%%%%%%%%%%%%%%%%%%%%%%%%%%%%%%%%%%%%%%%%%%%%%%%%
%Аннотация
%%%%%%%%%%%%%%%%%%%%%%%%%%%%%%%%%%%%%%%%%%%%%%%%%%%%%
\newcommand{\Rabstract}[1]{%
\centerline{%
\begin{minipage}{150mm}%
{%\setlength{\parindent}{4mm}
\small\sffamily%
#1}%
\end{minipage}}%
}

\newcommand{\Eabstract}[1]{%
\centerline{%
\begin{minipage}{150mm}%
{%\setlength{\parindent}{4mm}
\small\sffamily%
#1}%
\end{minipage}}%
}

%%%%%%%%%%%%%%%%%%%%%%%%%%%%%%%%%%%%%%%%%%%%%%%%%%%%%
%Ключевые слова
%%%%%%%%%%%%%%%%%%%%%%%%%%%%%%%%%%%%%%%%%%%%%%%%%%%%%
\newcommand{\Rkeywords}[1]{%
\vspace*{3mm}%
\centerline{%
\begin{minipage}{150mm}%
{%\setlength{\parindent}{4mm}
\small\sffamily%
\textit{Ключевые слова:} #1.}%
\end{minipage}}%
\vspace*{3mm}%
}

\newcommand{\Ekeywords}[1]{%
\vspace*{3mm}%
\centerline{%
\begin{minipage}{150mm}%
{%\setlength{\parindent}{4mm}
\small\sffamily%
\textit{Key words:} #1.}%
\end{minipage}}%
\vspace*{3mm}%
}

\newlength{\realparindent}%


%%%%%%%%%%%%%%%%%%%%%%%%%%%%%%%%%%%%%%%%%%%%%%%%%%%%%
%Библиографический список
%%%%%%%%%%%%%%%%%%%%%%%%%%%%%%%%%%%%%%%%%%%%%%%%%%%%%
\makeatletter
\renewcommand{\chapter}{\@startsection{chapter}{1}{0em}%
{3.5ex plus 1ex minus .2ex}{.9ex plus.2ex}%
{\zerosetcounter}}
%Стили нумерации формул, списка литературы
\addto\captionsrussian{%
\def\bibname{}%Меняем заголовок литературы
}%
\renewcommand{\@biblabel}[1]{#1.}%Оформление номера в списке литературы
\makeatother

\newenvironment{Rtwocolbib}
{%
\vspace*{3mm} %
\noindent
{\normalfont\bfseries\sffamily Библиографический список}%
\def\bibname{}
\small
%\begin{multicols}{2}%
\vspace*{-12mm}%
\begin{thebibliography}{99}
\setlength{\itemsep}{-4pt}
}{%
\end{thebibliography}
%\end{multicols}%
\normalsize}%

\newenvironment{Etwocolbib}
{%
\vspace*{1mm} %
\noindent
{\normalfont\bfseries\sffamily References}%
%\begin{otherlanguage}{english}
\small
%\begin{multicols}{2}%
%\vspace*{-45pt}%
\begin{enumerate}
\setlength{\itemsep}{-4pt}
}{%
\end{enumerate}
%\end{multicols}%
%\end{otherlanguage}
\normalsize
}%
%%%%%%%%%%%%%%%%%%%%%%%%%%%%%%%%%%%%%%%%%%%%%%%%%%%%%%%%%%%
\renewcommand{\arraystretch}{1.1}

%%%%%%%%%%%%%%%%%%%%%%%%%%%%%%%%%%%%%%%
%Благодарности
%%%%%%%%%%%%%%%%%%%%%%%%%%%%%%%%%%%%%%%
\renewcommand{\thanks}[1]{%
\vspace*{3mm}%
\noindent\textit{Благодарности. #1.}%
\vspace{2mm}%
}%

\newcommand{\ethanks}[1]{%
%\vspace*{3mm}%
\small
\noindent\textit{Acknowledgements: #1.}%
\vspace{3mm}%
\normalsize
}%

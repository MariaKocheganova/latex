\documentclass[10pt]{beamer}
\usepackage{amsmath}
\usepackage{amsfonts}
\usepackage[T2A]{fontenc}
\usepackage[russian]{babel}
\setbeamertemplate{caption}[numbered]
\usepackage[utf8]{inputenc}
% \usepackage[T2A]{fontenc}
%  \usepackage[cp1251]{inputenc}
% \usepackage[russian]{babel}
% \usepackage{amsmath}
% \input xy
% %\xyoption{all}
%  \usepackage{tikz}
%\newcommand*\circled[1]{\tikz[baseline=(char.base)]{
%            \node[shape=circle,draw,inner sep=2pt] (char) {#1};}}

\usepackage{beton}

\usepackage{tikz}
\newcommand{\Mark}{\{(\Gamma_i, \varkappa_i); \hm{} i \geqslant 0\}}
\newcommand*{\hm}[1]{#1\nobreak\discretionary{}%
	{\hbox{$\mathsurround=0pt #1$}}{}}% перенос арифметических знаков
\mode<presentation>{
	\usetheme{Warsaw}
	\usefonttheme{serif}
	\useoutertheme{infolines}
	
}
\renewcommand{\Pr}{{\mathbf P}}
\newcommand{\MarkThree}{\{(\Gamma_i, \varkappa_{3,i}); \hm{} i \geqslant 0\}}
% \title[fff\ldots{}]%
% {\textbf\large ааа} 
% \author{а} 
% \institute[1231]{аа}
% \date[аа 21.05.2018]{аа \\
% 	аа}




\begin{document}



\begin{frame}
	
	\begin{center}
		{\small 
		Национальный исследовательский Нижегородский\\
		государственный университет им.~Н.И.~Лобачевского
		}
		\bigskip
		
 		{
Кочеганов Виктор Михайлович			
			\medskip
			
			{\large
Моделирование и анализ тандема систем массового обслуживания с циклическим
управлением с продлением
}
			
 		}
		\vfill
		
		{\small 01.01.09~--- дискретная математика и математическая кибернетика}
		\medskip
		
		
		{\small Диссертация на соискание ученой степени\\ кандидата физико-математических наук }
		
	\end{center}
	
	\vfill
	
	
	\hfill 
	\begin{minipage}[h]{0.56\linewidth}\small
		Научный руководитель:\\
		д. ф.-м. н., доцент Зорин~А.В.
	\end{minipage}
	
	\vfill
	
	
	\centerline{\small Саратов, 2019}  
	
\end{frame}
  \begin{frame}{Прикладная мотивация}
      \begin{figure}[h]
    \centering
    \pgfimage[height=6cm]{highway.jpg}
    \caption{Современная развязка шоссе}
    \label{VK:fig:1}
  \end{figure}
\end{frame}
  \begin{frame}{Прикладная мотивация}
      \begin{figure}[h]
    \centering
    \pgfimage[height=6cm]{fff.jpg}
    \caption{Два автомобильных перекрестка}
    \label{VK:fig:1}
  \end{figure}
\end{frame}


\begin{frame}{Объекты и методы исследования}
Изучаемые управляемые объекты можно рассматривать как:
\begin{enumerate}
    \item {\it кибернетические системы} и применять методы математической кибернетики;
    \item {\it организационно-управленческие системы} и применять методы исследования операций;
    \item {\it системы массового обслуживания} и применять методы теории вероятностей и теории массового обслуживания.
\end{enumerate}
\end{frame}



\begin{frame}{Цели диссертации}
Целями данной работы являются:
\begin{enumerate}
    \item построение
и исследование математической модели тандема управляющих систем обслуживания по циклическому алгоритму с продлением;
    \item построение и исследование
 имитационной модели систем, осуществляющих циклическое управление с продлением тандемом перекрестков.
\end{enumerate}
\end{frame}

\begin{frame}{Сведения о публикациях}
Основные результаты по теме работы изложены в 11 работах, в том числе
    \begin{itemize}
        \item 3~--- в журналах, рекомендованных ВАК,
        \item 2~--- в библиографической базе Scopus, 
        \item 2~--- в библиографической базе Web of Science, 
        \item 11~--- в библиографической базе РИНЦ,
        \item 8~--- в тезисах докладов.
    \end{itemize}
    Получено 1 свидетельство о государственной регистрации программы для ЭВМ.
\end{frame}
\begin{frame}{Основные выступления на конференциях:}
  \begin{enumerate}
 \item Международная
научная конференция «Теория вероятностей, случайные процессы, математическая статистика и приложения» (Минск, Республика Беларусь, 2015 г.).
\item IX Международная конференция «Дискретные модели в теории управляющих систем» (Москва и Подмосковье, 2015 г.).
\item Международная
научная конференция «Distributed Computer and Communication
Networks. DCCN 2016» (Москва, 2016 г.).
\item XVIII Международная конференция «Проблемы теоретической кибернетики» (Пенза, 2017 г.).
\item XVI международная конференция имени А.Ф. Терпугова «Информационные
технологии и математическое моделирование» ИТММ-2017 (Казань, 2017 г.).
\item IX международная конференция по исследованию операций  (ORM-2018/ Москва, 2018 г.).
\item 4-я международная конференция по стохастическим методам (МКСМ-4/пос. Дивноморское, 2019 г.).
\end{enumerate} 
  \end{frame}
  
  \begin{frame}{Содержание по главам}
  \Large{\textbf{Глава 1. Построение вероятностной модели тандема}}\par


      \end{frame}
  
\begin{frame}{Содержательная постановка задачи}
        \begin{figure}[h]
    \centering
    \pgfimage[height=5cm]{Crossroads}
    \caption{Тандем перекрёстков}
    \label{VK:fig:1}
  \end{figure}
\end{frame}
























\begin{frame}{Тандем перекрестков как управляющая СМО}
  \begin{figure}[h]
    \centering
    \pgfimage[height=6cm]{SystemScheme}
    \caption{Структурная схема рассматриваемой СМО}
    \label{VK:fig:2}
  \end{figure}
\end{frame} 


\begin{frame}{Параметры системы}
\begin{itemize}
    \item 
$\color{blue}\lambda_1>0$, $\color{blue}\lambda_3>0$ --- интенсивности поступления групп требований по потокам  $\Pi_1$, $\Pi_3$ соответственно.
%Входные потоки $\Pi_1$ и $\Pi_3$ --- \textbf{неординарные пуассоновские потоки} групп требований с интенсивностями поступления групп требований по потоку $\lambda_1$ и $\lambda_3$ соотвественно.
  \item 
{\color{blue}Распределение числа заявок в группе} по потоку $\Pi_j$, $j \in \{1,3\}$, имеет производящую функцию:
$$
f_j(z) = \sum_{\nu=1}^{\infty} p_{\nu}^{(j)} z ^{\nu}, \quad |z|<(1+\varepsilon), \varepsilon>0.
$$
  \item 
$\color{blue}T^{(k,r)}>0$ --- неслучайное время нахождения обслуживающего устройства в состоянии $\Gamma^{(k,r)}$, $k\in \{0, 1, \ldots, d\}$, $r \in \{1, 2, \ldots, n_k\}$.  
  \item 
$\color{blue}\ell(k,r,j)\geqslant 0$ --- количество требований, содержащихся в потоке насыщения $\Pi^{\text{нас}}_j$ в состоянии  $\Gamma^{(k,r)}$.
  \item 
$\color{blue}L > 0$ --- порог числа требований в очереди $O_3$, при превышении которого начинается обслуживание очереди $O_3$.
\end{itemize}

\end{frame}

\begin{frame}[allowframebreaks]{Кибернетический подход}
Основные принципы кибернетического подхода:
  \begin{enumerate}
  \item \textbf{принцип дискретности} актов функционирования управляющей системы обслуживания во времени;
    \item \textbf{принцип совместного
    рассмотрения} поблочного строения управляющей системы
    обслуживания и ее функционирования во времени;
  \item \textbf{принцип нелокальности} при описании поблочного
    строения управляющей системы обслуживания.
  \end{enumerate}
  \framebreak
  
  
    \begin{figure}[h]
    \centering
    \pgfimage[height=2.5cm]{timings}
    \caption{Шкала моментов наблюдения}
    \label{VK:fig:3}
  \end{figure}
 
  \framebreak
  
  
Основными составляющими кибернетической системы являются:
  \begin{itemize}
  \item \textbf{схема}
  \begin{itemize}
      \item внешняя среда
      \item входные и выходные полюса
      \item внутренняя и внешняя память
      \item устройства по переработке информации во внутренней и внешней памяти
      \end{itemize}
  \item \textbf{информация} --- набор состояний среды, очередей в накопителях, обслуживающего устройства, потоков насыщения и потоков обслуженных требований
  \item \textbf{координата} блока --- номер блока на схеме
  \item \textbf{функция} --- обслуживание потоков по заданному алгоритму
  \end{itemize}
\end{frame}


% \begin{frame}{Кодирование информации}
% Пусть $Z_+$ --- множество целых неотрицательных чисел
%   \begin{itemize}
%   \item $\{e^{(1)}\}$ --- множество состояний \textbf{внешней среды} (одно состояние);
%   \item $Z^4_+$ --- множество состояний \textbf{входных полюсов};
%   \item $Z^4_+$ --- множество состояний \textbf{выходных полюсов};
%  \item $\Gamma=\{\Gamma^{(k,r)} \colon k=0,1,\ldots,d; r=1,2,\ldots n_k\}$ --- множество состояний \textbf{внутренней памяти};
%   \item $Z^4_+$ --- множество состояний \textbf{внешней памяти};
%   \item $\{r^{(1)}\}$ --- множество состояний \textbf{устройства по переработке информации во внешней памяти} (одно состояние)
%   \item граф переходов (будет описан ниже) описывает устройство по переработке информации во внутренней памяти
%   \end{itemize}
% \end{frame}

\begin{frame}{Необходимые случайные величины}
  \begin{itemize}
    \item $\tau_i \in {\mathbb R}_+$, $i=0$, $1$, \ldots --- момент смены состояния
    обслуживающего устройства;
    \item $\eta_{j,i} \in Z_+$ --- число требований потока $\Pi_j$, поступивших за
    промежуток $(\tau_i, \tau_{i+1}]$;
    \item $\xi_{j,i} \in Z_+$ --- число требований потока насыщения $\Pi^{\mbox{\scriptsize{нас}}}_j$ на промежутке $(\tau_i, \tau_{i+1}]$;
    \item $\varkappa_{j,i}$ --- число требований в
    очереди $O_j$ в момент $\tau_i$;
  \item $\Gamma_i\in\Gamma$ --- состояние обслуживающего устройства в момент $\tau_i$;
  \item $\overline{\xi}_{j,i} \in Z_+$ --- число требований
    выходного потока $\Pi^{\mbox{\scriptsize{вых}}}_j$ на промежутке
    $(\tau_i, \tau_{i+1}]$.
  \end{itemize}
  $j=1$,  $2$, $3$, $4$.
\end{frame}


\begin{frame}[allowframebreaks]{Граф переходов}
 Изменение состояний обслуживающего устройства: $\Gamma_{i+1}=h(\Gamma_i,\varkappa_{3,i})$.
   \begin{figure}[h]
    \centering
    \pgfimage[height=4.5cm]{SystemStates}
    \caption{Числовой пример. Времена}
    \label{VK:fig:3}
  \end{figure}
   \begin{figure}[h]
    \centering
    \pgfimage[height=6cm]{GraphScheme3}
    \caption{Числовой пример. Граф переходов (отображение $h(\cdot,\cdot)$).}
    \label{VK:fig:3}
  \end{figure}

  \begin{equation}
  \Gamma = \bigl( \bigcup_{k=1}^d C_k \bigr) \bigcup \{\Gamma^{(0,1)}, \Gamma^{(0,2)}, \ldots, \Gamma^{(0,n_0)}\}, \quad C_k = C_k^{\mathrm{I}} \cup C_k^{\mathrm{O}}  \cup C_k^{\mathrm{N}}.
  \end{equation}
  \begin{equation}
h(\Gamma^{(k,r)},y) = 
\begin{cases}
\Gamma^{(k,r\oplus_k 1)},& \quad \text{ если } \Gamma^{(k,r)}\in C_k\setminus C_k^{\mathrm{O}};\\
\Gamma^{(k,r\oplus_k 1)},& \quad \text{ если } \Gamma^{(k,r)}\in C_k^{\mathrm{O}} \text{ и } y>L;\\
\Gamma^{(0,h_1(\Gamma^{(k,r)}))},& \quad \text{ если } \Gamma^{(k,r)}\in C_k^{\mathrm{O}} \text{ и } y\leqslant L;\\
\Gamma^{(0,h_2(r))},& \quad \text{ если } k=0 \text{ и } y\leqslant L;\\
h_3(r),& \quad \text{ если } k=0 \text{ и } y > L,
\end{cases}
\end{equation}
где 
$$h_1(\cdot)\colon \bigcup_{k=1}^d C_k^{\mathrm{O}}\to N_0, \quad h_2(\cdot)\colon N_0\to N_0, \quad h_3(\cdot)\colon N_0 \to\bigcup_{k=1}^d C_k^{\mathrm{I}},$$ и $N_0=\{1,2, \ldots, n_0\}$.
Тогда 
\begin{equation}
\Gamma_{i+1} = h(\Gamma_i, \varkappa_{3,i}).
\end{equation}
\end{frame}



\begin{frame}{Функциональные соотношения }
Функционирование системы подчиняется следующим функциональным соотношениям:
\begin{equation}
\begin{aligned}
\overline{\xi}_{j,i}&=\min\{\varkappa_{j,i}+\eta_{j,i},\xi_{j,i}\}, \quad & j\in \{1,2,3\},\\
\varkappa_{j,i+1}&=\varkappa_{j,i}+\eta_{j,i}-\overline{\xi}_{j,i}, \quad & j\in \{1,2,3\},\\
\varkappa_{j,i+1}&=\max\{{0,\varkappa_{j,i}+\eta_{j,i}-\xi_{j,i}}\}, \quad & j\in \{1,2,3\},\\
\varkappa_{4,i+1}&=\varkappa_{4,i}+\eta_{4,i}-\eta_{2,i}, \quad &\\
\xi_{4,i} & = \varkappa_{4,i}, & \\
\eta_{4,i} & = \min\{ \varkappa_{1,i} + \eta_{1,i}, \xi_{1,i}\}.
\end{aligned}
\label{rekk}
\end{equation}
\end{frame}




\begin{frame}[allowframebreaks]{Свойства условных распределений}
Определим функции $\varphi_j(\cdot,\cdot)$, $j\in \{1,3\}$, и $\psi(\cdot, \cdot, \cdot)$ из разложений:
\begin{equation*}
\sum_{\nu=0}^{\infty} z^\nu\varphi_j(\nu,t) = \exp\{\lambda_j t (f_j(z)-1)\}, \quad \psi(k;y,u)=C_y^k u^k (1-u)^{y-k}.	
\end{equation*}

Пусть $a=(a_1, a_2, a_3, a_4) \in \mathbb{Z}_+^4$ и $x=(x_1, x_2, x_3, x_4) \in \mathbb{Z}_+^4$.
%\begin{block}{Предположение 1}

Тогда вероятность $\varphi(a,k,r,x)$ одновременного выполнения равенств $\eta_{1,i}=a_1$, $\eta_{2,i}=a_2$, $\eta_{3,i}=a_3$, $\eta_{4,i}=a_4$ при условии  $\nu_i=(\Gamma{(k,r)}; x)$ есть 
\begin{equation}
\!\!\varphi_1(a_1,h_T(\Gamma^{({k},{r})},x_3)) \times \psi(a_2,x_4, p_{\tilde{k},\tilde{r}}) \times \varphi_3(a_3,h_T(\Gamma^{({k},{r})},x_3))
\times \delta_{a_4,\min{\{\ell(\tilde{k},\tilde{r},1), x_1+a_1}\}},
\label{prob:1}
\end{equation}
%\end{block}
где
\begin{equation*}
\Gamma^{(\tilde{k},\tilde{r})}=h(\Gamma^{(k,r)},x_3), \quad \delta_{i,j}=\begin{cases} 1, \quad \text{ если }i=j,\\0, \quad \text{ если } i\neq j,
\end{cases}
\end{equation*}
и 
$$
T_{i+1}=h_T(\Gamma_i,\varkappa_{3,i})= T^{(k,r)},\quad  \Gamma^{(k,r)}=\Gamma_{i+1}=h(\Gamma_i,\varkappa_{3,i}).
$$
\framebreak

Пусть $b=(b_1, b_2, b_3, b_4) \in \mathbb{Z}_+^4$. 

Тогда вероятность $\zeta(b, k, r, x)$ одновременного выполнения равенств $\xi_{1,i}=b_1$, $\xi_{2,i}=b_2$, $\xi_{3,i}=b_3$, $\xi_{4,i}=b_4$ при фиксированном значении метки $\nu_i=(\Gamma{(k,r)}; x)$ есть
\begin{equation}
\delta_{b_1,\ell(\tilde{k},\tilde{r},1)} \times \delta_{b_2,\ell(\tilde{k},\tilde{r},2)} \times 
\delta_{b_3,\ell(\tilde{k},\tilde{r},3)} \times \delta_{b_4,x_4}.
\label{prob:2}
\end{equation}
где $\tilde{k}$ и $\tilde{r}$ такие, что $\Gamma^{(\tilde{k},\tilde{r})}=h(\Gamma^{(k,r)},x_3)$.
\end{frame}


\begin{frame}[allowframebreaks]{Результаты первой главы}

    {\bf Теорема 1.1} {\it 
    Пусть 
    $$\gamma_0=\Gamma^{(k_0,r_0)} \in \Gamma, \quad x^0=(x_{1,0},x_{2,0}, x_{3,0},x_{4,0})\in \mathbb{Z}_+^4$$ фиксированы. Тогда существует вероятностное пространство $(\Omega, {\cal F}, {\mathbf P}(\cdot))$ и заданные на нем случайные величины $\eta_{j,i}=\eta_{j,i}(\omega)$, $\xi_{j,i}=\xi_{j,i}(\omega)$, $\overline{\xi}_{j,i}=\xi_{j,i}(\omega)$, $\varkappa_{j,i}=\varkappa_{j,i}(\omega)$ и случайные элементы $\Gamma_i=\Gamma_i(\omega)$, $i\geqslant 0$, $j\in \{1, 2, 3, 4\}$, такие, что 
    \begin{itemize}
    \item имеют место равенства $\Gamma_0(\omega) = \gamma_0$ и $\varkappa_0(\omega)=x^0$;
    \item выполняются соотношения \eqref{rekk}~-- \,\eqref{prob:2};
    \item для любых  $a$, $b$, $x^t=(x_{1,t},x_{2,t},x_{3,t},x_{4,t}) \in \mathbb{Z}_+^4$, $\Gamma^{(k_t,r_t)} \in \Gamma$, $t = 1, 2, \ldots$, условное распределение векторов $\eta_i$, и $\xi_i$ имеет вид 
\begin{multline*}
    {\mathbf P}(\{ \omega \colon \eta_i = a, \xi_i=b\} |\cap_{t=0}^{i}\{\omega\colon \Gamma_t=\Gamma{(k_t,r_t)}, \varkappa_t=x^t\})=\\
=\varphi(a,k_i,r_i,x^i)\times \zeta(b,k_i,r_i,x^i).
\end{multline*}
    \end{itemize}
}

\framebreak

Результаты 1-й главы:
\begin{enumerate}
    \item задача управления тандемом перекрестков поставлена на содержательном уровне;
    \item построена строгая математическая модель в виде вероятностного пространства с определенными на нем случайными величинами и элементами.

\end{enumerate}


\end{frame}



  \begin{frame}{Содержание по главам}
  \Large{\textbf{Глава 2. Анализ кибернетической системы как стохастической последовательности}}\par


      \end{frame}

\begin{frame}{Стохастическая последовательность, описывающая тандем}
В главе исследуются свойства общей последовательности
\begin{equation}
    \{(\Gamma_i, \varkappa_{1,i},\varkappa_{2,i}, \varkappa_{3,i}, \varkappa_{4,i}); i \geqslant 0\}
\end{equation}
        \begin{figure}[h]
    \centering
    \pgfimage[height=4.5cm]{Crossroads}
    \caption{Тандем перекрёстков}
    \label{VK:fig:1}
  \end{figure}
\end{frame}


\begin{frame}{Марковское свойство}
\begin{block}
    {\bf Теорема 2.1} {\it 
Пусть $\Gamma_0=\Gamma^{(k,r)}\in \Gamma$ и $(\varkappa_{1,0},\varkappa_{2,0},\varkappa_{3,0},\varkappa_{4,0})=(x_1,x_2,x_3,x_4)\in \mathbb{Z}_+^4$ фиксированы. 

Тогда последовательность $\{(\Gamma_i, \varkappa_{1,i},\varkappa_{2,i}, \varkappa_{3,i}, \varkappa_{4,i}); i \geqslant 0\}$ является однородной счетной цепью Маркова.
}
\end{block}
\end{frame}


\begin{frame}[allowframebreaks]{Переходные вероятности}
Введем множества ${\mathbb A}_{\mathrm{trans}}(x,\tilde{x},\tilde{k},\tilde{r})$  следующим образом:
\begin{align}
{\mathbb A}_{\mathrm{trans}}(x,\tilde{x},\tilde{k},\tilde{r}) &= {\mathbb A}_{\mathrm{trans}}^0(x,\tilde{x},\tilde{k},\tilde{r}) \cap {\mathbb A}_{\mathrm{trans}}^1(x,\tilde{x},\tilde{k},\tilde{r})\cap {\mathbb A}_{\mathrm{trans}}^2(x,\tilde{x},\tilde{k},\tilde{r}),\label{A:trans:1}\\
{\mathbb A}_{\mathrm{trans}}^0(x,\tilde{x},\tilde{k},\tilde{r}) &= \{(a_1,a_2) \in \mathbb{Z}_+^2 \colon a_2 = \min{\{\ell(\tilde{k},\tilde{r},1), x_1+a_1}\} +x_4-\tilde{x}_4\},\\
{\mathbb A}_{\mathrm{trans}}^1(x,\tilde{x},\tilde{k},\tilde{r}) &= \{(a_1,a_2) \in \mathbb{Z}_+^2 \colon \tilde{x}_1=\max{\{0,x_1+a_1-\ell(\tilde{k},\tilde{r},1)\}}\},\\
{\mathbb A}_{\mathrm{trans}}^2(x,\tilde{x},\tilde{k},\tilde{r}) &= \{(a_1,a_2) \in \mathbb{Z}_+^2 \colon  \tilde{x}_2=\max{\{0,x_2+a_2-\ell(\tilde{k},\tilde{r},2)\}}\}.\label{A:trans:2}
\end{align}
\framebreak 

 {\bf Теорема 2.3}
{\it 
Пусть $x$, $\tilde{x}\in \mathbb{Z}_+^4$ и $\Gamma^{(k,r)}$, $\Gamma^{(\tilde{k},\tilde{r})}=h(\Gamma^{(k,r)},x_3) \in \Gamma$. Тогда переходные вероятности однородной счетной марковской цепи $\Mark$ вычисляются по следующей формуле:
\begin{multline}
\Pr (\{\omega\colon \Gamma_{i+1}=\Gamma^{(\tilde{k},\tilde{r})},\varkappa_{i+1}=\tilde{x} \}| \{\omega\colon \Gamma_{i}=\Gamma^{(k,r)},\varkappa_i=x\})=\\ 
=\widetilde{\varphi}_3(\tilde{k},\tilde{r},h_T(\Gamma^{(k,r)},x_3),x_3,\tilde{x}_3)\times \\ \times
\sum_{(a_1,a_2)\in {\mathbb A}_{\mathrm{trans}}}\varphi_1(a_1,h_T(\Gamma^{(k,r)},x_3))  \psi(a_2,x_4, p_{\tilde{k},\tilde{r}}).
\label{transitionToProve}
\end{multline}
}

\end{frame}



\begin{frame}{Сообщающиеся состояния}
    Для классификации состояний необходимо определить взаимную достижимость состояний цепи  $\Mark$.
    
    Например, сообщаются ли состояния
    вида $$(\Gamma^{(0,r_0)},x^0), \quad r_0=\overline{1,n_0}, \quad x^0 \in \mathbb{Z}_+^4,\quad x_{3,0} \leqslant L$$
   с состояниями вида   $$(\Gamma^{(0,\tilde{r})},(0,0,\tilde{x}_3,0)), \quad \tilde{r} = \overline{1,n_0}, \quad \tilde{x}_3\geqslant x_{3,0}.$$
    
    
 {\bf Лемма 2.1}
{\it 
Для любых состояний $(\Gamma^{(0,r_0)},x^0)$, $r_0=\overline{1,n_0}$, $x^0 \in \mathbb{Z}_+^4$, $x_{3,0} \leqslant L$, и $(\Gamma^{(0,\tilde{r})},(0,0,\tilde{x}_3,0))$, $\tilde{r} = \overline{1,n_0}$, $\tilde{x}_3\geqslant x_{3,0}$, существует такое натуральное число $N$, что 
\begin{equation*}
\Pr(\{\omega\colon\Gamma_{N}=\Gamma^{(0,\tilde{r} )}, \varkappa_{N}=(0,0,\tilde{x}_3,0)\}|
\{\omega\colon\Gamma_{0}=\Gamma^{(0,r_0)}, \varkappa_{0}=x^0\})>0.
\end{equation*}
}
\end{frame}


\begin{frame}[allowframebreaks]{Результаты второй главы}
    Леммы 2.1~-- 2.9 подводят к основному результату главы 2.
    
 {\bf Теорема 2.5}
{\it 
Состояния вида
$(\Gamma^{(\tilde{k},\tilde{r})},\tilde{x})$,
где $\tilde{k}=\overline{0,d}$, $\tilde{r} = \overline{1,n_{\tilde{k}}}$, $\tilde{x}\in \mathbb{Z}_+^4$,
\begin{gather}
(\tilde{x}_1>0) \Rightarrow (\tilde{x}_4\geqslant \ell(\tilde{k},\tilde{r},1)),\label{reachable:1}\\
\tilde{x}_3\geqslant \max{\Bigl\{0,L+1-\sum_{s=1}^{\tilde{r}}\ell(k,s,3)\Bigr\}}, \text{ если } \tilde{k}>0,\\
\tilde{x}_3\geqslant \max{\Big\{0,L+1-\max_{k=\overline{1,d}}{\{\sum_{s=1}^{n_{\tilde{k}}} \ell(\tilde{k},s,3)\}}\Bigr\}}, \text{ если } \tilde{k}=0,\label{reachable:2}
\end{gather}
 и только они являются существенными.
 \label{important:states:basic}
}
\framebreak

Результаты 2-й главы:
\begin{enumerate}
    \item строго доказано марковское свойство последовательности $\Mark$;
    \item проведена классификация состояний по арифметическим свойствам марковской цепи $\Mark$.

\end{enumerate}


\end{frame}




  \begin{frame}{Содержание по главам}
  \Large{\textbf{Глава 3. Анализ первичных и промежуточной очередей системы}}\par


      \end{frame}
\begin{frame}{Тандем}
        \begin{figure}[h]
    \centering
    \pgfimage[height=4.5cm]{Crossroads}
    \caption{Тандем перекрёстков}
    \label{VK:fig:1}
  \end{figure}
\end{frame}

        \begin{frame}{Промежуточная очередь}
        \begin{block}
 {\bf Теорема 3.1}
{\it 
Для того, чтобы последовательность $\{E\varkappa_{4,i}(\omega); i =0, 1, \ldots\}$ была ограничена, достаточно выполнения неравенства
\begin{equation*}
   % \min_{\substack{k=\overline{1,d}\\ j=1,3}} {\{p_{k,r}\}} > 0.
    \min_{k=\overline{0,d}, r=\overline{1,n_k}} {\{p_{k,r}\}} > 0.
\end{equation*}
}
\end{block}
      \end{frame}


        \begin{frame}[allowframebreaks]{Низкоприоритетная очередь}
При помощи итеративно-мажорантного метода были получены достаточное и необходимое условия существования стационарного распределения цепи $$\MarkThree:$$
 {\bf Теорема 3.4}
{\it 
Для того, чтобы марковская цепь $\MarkThree$ имела стационарное распределение $Q(\gamma,x)$, $(\gamma,x)\in \Gamma \times {\mathbb Z}_+$, достаточно выполнения неравенства 
\begin{equation}
\min_{k=\overline{1,d}} { \frac{\sum_{r = 1}^{n_k} \ell(k,r,3) }{\lambda_3 f_3'(1) \sum_{r=1}^{n_k} T^{(k,r)} }}>1.
\label{sufficient:low}
\end{equation}
}
\framebreak

Необходимое условие существования стационарного распределения для марковской цепи $\MarkThree$ 

 {\bf Теорема 3.5}
{\it 
Для того, чтобы марковская цепь $\MarkThree$ имела стационарное распределение $Q_3(\gamma,x)$, $(\gamma,x)\in \Gamma \times {\mathbb Z}_+$, необходимо выполнение неравенства
$$
\max_{k=\overline{1,d}} { \frac{\sum_{r = 1}^{n_{k}}\ell(k,r,3)}{\lambda_3 f_3'(1) \sum_{r = 1}^{n_k} T^{(k,r)}} } >1.
$$
}

\end{frame}

\begin{frame}{Очереди первичных требований}
Очередями первичных требований будем называть очереди $O_1$ и $O_3$.
\vfill

 {\bf Теорема 3.9}
{\it 
Для того, чтобы марковская цепь $\{(\Gamma_i, \varkappa_{1,i},\varkappa_{3,i}); i \geqslant 0\}$ имела стационарное распределение $Q_1(\gamma,x_1,x_3)$, $(\gamma,x_1,x_3)\in \Gamma \times {\mathbb Z}^2_+$, достаточно выполнения неравенств
\begin{equation}
\min_{k=\overline{0,d}} { \frac{\sum_{r = 1}^{n_k} \ell(k,r,1) }{\lambda_1 f_1'(1) \sum_{r=1}^{n_k} T^{(k,r)} }}>1, \quad 
\min_{k=\overline{1,d}} { \frac{\sum_{r = 1}^{n_k} \ell(k,r,3) }{\lambda_3 f_3'(1) \sum_{r=1}^{n_k} T^{(k,r)} }}>1.
\label{sufficient:double}
\end{equation}

}
\end{frame}


\begin{frame}{Результаты третьей главы}
Результаты 3-й главы:
\begin{enumerate}
    \item найдено достаточное условие ограниченности последовательности $$\{E\varkappa_{4,i}(\omega); i =0, 1, \ldots\},$$ характеризующей промежуточную очередь $O_4$;
    \item найдено достаточное и необходимое условия существования стационарного распределения для марковской цепи $\MarkThree$;
    \item найдено достаточное условие существования стационарного распределения марковской цепи $\{(\Gamma_i, \varkappa_{1,i},\varkappa_{3,i}); i \geqslant 0\}$.
\end{enumerate}

\end{frame}


  \begin{frame}{Содержание по главам}
  \Large{\textbf{Глава 4. Исследование системы управления тандемом с помощью имитационной модели}}\par
\end{frame}

\begin{frame}{Квазиоптимальное управление системой}
    Управляющие параметры:
    
    \begin{itemize}
        \item длительности $T_3$ и $T_4$ сигналов второго светофора;
        \item длительность $T_{prol}$ сигнала продления второго светофора;
        \item порог продления $L$.
    \end{itemize}
    
    Показатель качества функционирования системы: $E \gamma$~--- среднее взвешенное время пребывания произвольного требования.
    
    Для определения квазистационарного режима запускаются несмещенная (<<0>>) и смещенная (<<+>>) системы: с пустыми и непустыми начальными очередями соответственно. 
\end{frame}


\begin{frame}{Квазистационарный режим}
    В конце каждого такта считаются значения
\begin{equation}
   \gamma_{j,\cdot}^0 = \frac{1}{\tilde{\mathcal{V}}_j^0}\sum_{\nu} \gamma_{j,\nu}^0, \quad \gamma_{j,\cdot}^+ = \frac{1}{\tilde{\mathcal{V}}_j^+}\sum_{\nu} \gamma_{j,\nu}^+ 
\end{equation}

Кроме того, для несмещенной системы  считаются входящий и выходящий потоки:
\begin{equation}
    F^{0}_{\text{in},1} = \sum_n \alpha^{0}_{\text{in},1,n}, \quad 
    F^{0}_{\text{out},4} = \sum_n \alpha^{0}_{\text{out},4,n},
\end{equation}
\begin{equation}
    F^{0}_{\text{in},3} = \sum_n \alpha^{0}_{\text{in},3,n}, \quad 
    F^{0}_{\text{out},3} = \sum_n \alpha^{0}_{\text{out},3,n}.
\end{equation}


Стационарный режим считается достигнутым, если выполнены все неравенства:
\begin{equation}
    \frac{|\gamma_{j,\cdot}^0 - \gamma_{j,\cdot}^+|}{\gamma_{j,\cdot}^0} < \delta_1, \quad
    \frac{F^{0}_{\text{in},1}}{F^{0}_{\text{out},4}} < \delta_2, \quad 
    \frac{F^{0}_{\text{in},3}}{F^{0}_{\text{out},3}} < \delta_3.
\end{equation}
\end{frame}


\begin{frame}{Показатель качества функционирования системы}
     $\hat{E}\gamma_{j}=\frac{1}{\tilde{\mathcal{V}}_j}\sum_{\nu}\zeta_{j,\nu}$  -- оценка математического ожидания времени пребывания в системе произвольного требования потока $\Pi_j$.
     \vfill
     Результирующая оценка целевой функции:
\begin{equation}
 \hat{E}\gamma=\frac{\sum_{j\in\{1,3\}} (\lambda_j \sum_{\nu\geqslant1}\nu p_{\nu}^{(j)})\hat{E}\gamma_{j} }{\sum_{j\in\{1,3\}} \lambda_j \sum_{\nu\geqslant1}\nu p_{\nu}^{(j)}}.
\end{equation}
\end{frame}


\begin{frame}[allowframebreaks]{Результаты экспериментов}
\begin{columns}
    \begin{column}{0.48\textwidth}
        \begin{figure}[h]
    \centering
    \pgfimage[height=4.5cm]{target_thrs_-1.png}
    \caption{Циклический алгоритм}
    \label{VK:fig:1}
  \end{figure}
    \end{column}
    \begin{column}{0.48\textwidth}
        \begin{figure}[h]
    \centering
    \pgfimage[height=4.5cm]{target_thrs_15.png}
    \caption{Алгоритм с продлением}
    \label{VK:fig:1}
  \end{figure}
    \end{column}
\end{columns}
\framebreak

\begin{columns}
    \begin{column}{0.33\textwidth}
        \begin{figure}[h]
    \centering
    \pgfimage[height=4cm]{target_thrs_5.png}
    \caption{L=5}
    \label{VK:fig:1}
  \end{figure}
    \end{column}
    \begin{column}{0.33\textwidth}
        \begin{figure}[h]
    \centering
    \pgfimage[height=4cm]{target_thrs_10.png}
    \caption{L=10}
    \label{VK:fig:1}
  \end{figure}
    \end{column}
        \begin{column}{0.33\textwidth}
        \begin{figure}[h]
    \centering
    \pgfimage[height=4cm]{target_thrs_15.png}
    \caption{L=15}
    \label{VK:fig:1}
  \end{figure}
    \end{column}
\end{columns}


\end{frame}



\begin{frame}{Результаты четвертой главы}
Результаты 4-й главы:
\begin{enumerate}
    \item реализована имитационная модель тандема перекрестков;
    \item разработан метод определения момента достижения системой квазистационарного режима;
    \item проведен анализ имитационной модели и, в частности, подтверждены и расширены результаты теоретических исследований работы.
\end{enumerate}

\end{frame}



\begin{frame}{Положения, выносимые на защиту}
    \begin{enumerate}

\item Методика построения вероятностного пространства для тандема систем с немгновенным перемещением требований между ними.
\item Методика нахождения условий существования стационарного режима в системах управления потоками неоднородных требований с циклическим алгоритмом и алгоритмом с продлением.
\item Метод определения момента достижения управляемой системой обслуживания квазистационарного режима.
    \end{enumerate}
\end{frame}

\begin{frame}{Основные результаты}
    \begin{enumerate}
        \item построена строгая математическая модель тандема с циклическим алгоритмом управления и алгоритмом с продлением;
        \item найдены условия существования стационарного распределения для цепи, включающей низкоприоритетную очередь;
        \item найдены условия существования стационарного распределения для цепей, включающих промежуточную и первичные очереди;
        \item разработана имитационная модель для изучения исходной системы;
        \item на основе имитационной модели были подтверждены и расширены результаты, полученные теоретически.
    \end{enumerate}
\end{frame}

\begin{frame}[allowframebreaks]{Основные публикации:}
    \begin{enumerate}
    
   \item Кочеганов~В.М., Зорин~А.В. Вероятностная модель тандема систем массового обслуживания с циклическим управлением с продлением // Теория вероятностей, случайные процессы, математическая статистика и приложения: материалы Междунар. науч. конф., посвящ. 80-летию проф., д-ра физ.-мат. наук Г.А. Медведева, Минск, 23–26 февр. 2015.~-- С.~94–99.
\item Кочеганов~В.М., Зорин~А.В. Дискретная модель колебания длины низкоприоритетной очереди в тандеме систем массового обслуживания при циклическом алгоритме с продлением // Дискретные модели в теории управляющих систем: IX Международная конференция, Москва и Подмосковье, 20–22 мая 2015.~-- С.~126–129.
\item Kocheganov~V.M., Zorine~A.V. Low-Priority Queue Fluctuations in Tandem of Queuing Systems Under Cyclic Control with Prolongations // Distributed Computer and Communication Networks. Ser. Communications in Computer and Information Science. 2016. V.~601. P.~268-279.



\item Кочеганов~В.М., Зорин~А.В. Достаточное условие существования стационарного режима низкоприоритетной очереди в тандеме систем массового обслуживания // Вестник Волжской государственной академии водного транспорта. 2017. Выпуск~50. С.~47–55.

% \item
% Kocheganov~V.M., Zorine~A.V. Low-Priority Queue Fluctuations in Tandem of Queuing Systems Under Cyclic Control with Prolongations // DCCN. Ser. Communications in Computer and Information Science. 2016. V.~601. P.~268--279.

\item
Kocheganov~V., Zorine~A. Primary input flows in a tandem under prolongable cyclic service // DCCN. 2017. Pp.~517--525.

\item Кочеганов~В.М., Зорин~А.В. Достаточное условие существования стационарного режима очередей первичных требований в тандеме систем массового обслуживания // Вестник ТвГУ. Серия: Прикладная математика. 2018. №~2. С.~49~--74.

%         Кочеганов В.М., Зорин А.В. Вероятностная модель тандема систем массового обслуживания с циклическим управлением с продлением // Теория вероятностей, случайные процессы, математическая статистика и приложения: материалы Междунар. науч. конф., посвящ. 80-летию проф., д-ра физ.-мат. наук Г.А. Медведева, Минск, 23–26 февр. 2015 г. / редкол. Н.Н. Труш [и др.]. — Минск: РИВШ, 2015. — С. 94–99.
% Тип: статья, печатная
% 0.693 п.л.
% http://elib.bsu.by/bitstream/123456789/111720/1/%D0%9A%D0%BE%D1%87%D0%B5%D0%B3%D0%B0%D0%BD%D0%BE%D0%B2.pdf


% Кочеганов В.М., Зорин А.В. Дискретная модель колебания длины низкоприоритетной очереди в тандеме систем обслуживания при циклическом алгоритме с продлением // IX Международная конференция "Дискретные модели в теории управляющих систем": Москва и Подмосковье, 20-22 мая 2015 г.: Труды / Отв. ред. В.Б. Алексеев, Д.С. Романов, Б.Р. Данилов. - М.: МАКС Пресс, 2015. С. 126-128.
% Тип: статья, печатная
% 0.3465 п.л.
% http://agora.guru.ru/display.php?conf=dm9&page=item001&PHPSESSID=6120k7rjdm4rb4lv5hakfueut6

   
% Kocheganov V.M., Zorine A.V. Low-Priority Queue Fluctuations in Tandem of Queuing Systems Under Cyclic Control with Prolongations / Distributed Computer and Communication Networks. Ser. Communications in Computer and Information Science, 2015. – V. 601. – P. 268–279
% https://link.springer.com/chapter/10.1007/978-3-319-30843-2_28


% Kocheganov V.M., Zorine A.V. Low-priority queue and server's steady-state existence in a tandem under prolongable cyclic service // Distributed Computer and Communication Networks. DCCN 2016. Communications in Computer and Information Science, 2016. – V. 678. – P. 210-221.
% https://link.springer.com/chapter/10.1007/978-3-319-51917-3_19


% Кочеганов В.М., Зорин А.В. Достаточное условие существования стационарного режима низкоприоритетной очереди в тандеме систем массового обслуживания // Вестник волжской государственной академии водного транспорта, 2017, N. 50. — С. 47–55. (РИНЦ)

% Кочеганов В.М., Зорин А.В. Изучение процесса управления потоками первичных требований в тандеме систем обслуживания с циклическим алгоритмом с продлением // Проблемы теоретической кибернетики: XVIII международная конференция (Пенза, 19–23 июня 2017 г.) : Материалы : Под редакцией Ю. И. Журавлева. — М. : МАКС Пресс, 2017. — С. 135–137.


% Кочеганов В.М., Зорин А.В. Анализ потоков первичных требований в тандеме при циклическом управлении с продлением // Информационные технологии и математическое моделирование (ИТММ-2017): Материалы XVI Международной конференции имени А.Ф. Терпугова (29 сентября – 3 октября 2017 г.) — Томск: Изд-во НТЛ, 2017 — Ч. 1. — С. 81–87.


% Kocheganov V., Zorine A. Primary input flows in a tandem under prolongable cyclic service // Распределенные компьютерные и телекоммуникационные сети: управление, вычисление, связь (DCCN-2017) : Материалы Двадцатой мужденар. науч. конфер., 25–29 сент. 2017 г. Москва: / Ин-т прболем упр. им. В.А.Трапезникова Рос. акад. наук; под общ. ред. В.М.Вишневского. — М.: ТЕХНОСФЕРА, 2017. — C. 517–525.
% Кочеганов В.М., Зорин А.В. Достаточное условие существования стационарного режима очередей первичных требований в тандеме систем массового обслуживания // Вестник ТвГУ. Серия: Прикладная математика. 2018. № 2. С. 49-74. https://doi.org/10.26456/vtpmk193


    \end{enumerate}
\end{frame}
\begin{frame}
\Huge{\centerline{\color{blue} Благодарю}
\centerline{\color{blue} 
за внимание!}}
\end{frame}

\appendix
\section{Приложение}


\end{document}



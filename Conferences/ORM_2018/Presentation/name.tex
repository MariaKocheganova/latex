\documentclass[10pt]{beamer}

\usepackage{color}
\usepackage[T2A]{fontenc}
\usepackage[russian]{babel}
\usepackage{amsmath}
\newcommand{\No}{\textnumero}
\setbeamertemplate{caption}[numbered]
\usepackage[figurename=Fig.]{caption}


\usepackage{beton}

\usepackage{tikz}
\usepackage[utf8]{inputenc}

\defbeamertemplate{footline}{centered page number}
{%
  \hspace*{\fill}%
  \usebeamercolor[fg]{page number in head/foot}%
  \usebeamerfont{page number in head/foot}%
  \insertpagenumber\,/\,\insertpresentationendpage%
  \hspace*{\fill}\vskip2pt%
}
\setbeamertemplate{footline}[centered page number]
%\mode<presentation>{
%\usetheme{Rochester}
%}
\newcommand{\backupbegin}{
   \newcounter{framenumberappendix}
   \setcounter{framenumberappendix}{\value{framenumber}}
}
\newcommand{\backupend}{
   \addtocounter{framenumberappendix}{-\value{framenumber}}
   \addtocounter{framenumber}{\value{framenumberappendix}} 
}


\mode<presentation>{
%  \usetheme{Madrid}
  \usetheme{AnnArbor}
  \usefonttheme{serif}
}
\makeatletter
\setbeamertemplate{footline}
{
  \leavevmode%
  \hbox{%
  \begin{beamercolorbox}[wd=.333333\paperwidth,ht=2.25ex,dp=1ex,center]{author in head/foot}%
    \usebeamerfont{author in head/foot}\insertshortauthor%~~\beamer@ifempty{\insertshortinstitute}{}{(\insertshortinstitute)}
  \end{beamercolorbox}%
  \begin{beamercolorbox}[wd=.333333\paperwidth,ht=2.25ex,dp=1ex,center]{title in head/foot}%
    \usebeamerfont{title in head/foot}\insertshorttitle
  \end{beamercolorbox}%
  \begin{beamercolorbox}[wd=.333333\paperwidth,ht=2.25ex,dp=1ex,right]{date in head/foot}%
    \usebeamerfont{date in head/foot}\insertshortdate{}\hspace*{2em}
    \insertframenumber{} / \inserttotalframenumber\hspace*{2ex} 
  \end{beamercolorbox}}%
  \vskip0pt%
}
\makeatother


\begin{document}

\title[ASYMPTOTIC PROPERTIES OF ...]{\normalsize \color{blue} \MakeUppercase{Asymptotic properties of tandem service and control operation in a class of cyclic algorithms with prolongation}}

\author[V.M.~Kocheganov, A.V.~Zorine (NNSU)]{\textbf{V.M.~Kocheganov}, A.V.~Zorine}
\institute[NNSU]{\normalsize N.I.~Lobachevsky National Research \\ State University of Nizhny Novgorod }
\date[25-29.09.2017]{
ORM-2018 Germeyer-100 \\ 
 22-27.10.2018, Moscow
}

\begin{frame}
\titlepage
\end{frame}

\begin{frame}{Motivation}
      \begin{figure}[h]
    \centering
    \pgfimage[height=6cm]{fff.jpg}
    \caption{Two adjacent crossroads}
    \label{VK:fig:1}
  \end{figure}
\end{frame}

\begin{frame}{Related research}
    \begin{itemize}
    \item \textbf{Yamada~K., Lam~T.N.} Simulation analysis of two adjacent traffic signals // Proceedings of the 17th winter simulation conference.~--- New York: ACM, 1985.~--- Pp.~454--464.
    \item \textbf{Afanasyeva~L.G., Bulinskaya~E.V. } Mathematical models of transport systems based on queueing theory~// Works of Moscow Institute of Physics and Technology.~--- 2010.~--- \No{4}.~--- Pp.6--21. 
\item \textbf{Zorine~A.V.} Stability of a tandem of queueing systems with Bernoulli noninstantaneous transfer of  customers~// Theory of Probability and Mathematical Statistics.~--- 2012.~--- V.~84.~--- Pp.~173--188.
\item \textbf{Kuznetsov~N.Yu.} A heuristic algorithm to control conflicting nonstationary traffic flows~// Cybernetics and Systems Analysis.~--- 2018. V.~54.~--- Pp.~707--715.
    \end{itemize}
\end{frame}


\begin{frame}{Tandem of crossroads }
  \begin{figure}[h]
    \centering
    \pgfimage[height=5cm]{Crossroads}
    \caption{Tandem of crossroads }
    \label{VK:fig:1}
  \end{figure}
\end{frame} 


\begin{frame}{Cybernetic control system}
  \begin{figure}[h]
    \centering
    \pgfimage[height=6cm]{SystemScheme}
    \caption{Cybernetic control system scheme}
    \label{VK:fig:2}
  \end{figure}
\end{frame} 


\begin{frame}{System parameters}
\begin{itemize}
    \item 
$\color{blue}\lambda_1>0$, $\color{blue}\lambda_3>0$ --- arrival intensities of customer groups by input flows  $\Pi_1$, $\Pi_3$ respectively.
%Входные потоки $\Pi_1$ и $\Pi_3$ --- \textbf{неординарные пуассоновские потоки} групп требований с интенсивностями поступления групп требований по потоку $\lambda_1$ и $\lambda_3$ соотвественно.
  \item 
{\color{blue}Number of customers in a group distribution} by input flow $\Pi_j$, $j \in \{1,3\}$, has generating function:
\begin{equation}
f_j(z) = \sum_{\nu=1}^{\infty} p_{\nu}^{(j)} z ^{\nu}, \quad |z|<(1+\varepsilon), \varepsilon>0.
    \end{equation}
  \item 
$\color{blue}T^{(k,r)}>0$ --- deterministic time for server to be in state $\Gamma^{(k,r)}$, $k\in \{0, 1, \ldots, d\}$, $r \in \{1, 2, \ldots, n_k\}$.  
  \item 
$\color{blue}\ell(k,r,j)\geqslant 0$ --- number of customers in saturation flow $\Pi^{\text{sat}}_j$  during state  $\Gamma^{(k,r)}$.
  \item 
$\color{blue}L > 0$ --- customers number threshold in queue $O_3$, being exceeded queue $O_3$ begins to be serviced.
\end{itemize}

\end{frame}

\begin{frame}{Cybernetic approach}
Basic principles of cybernetic approach:
\medskip
  \begin{enumerate}
  \item  \textbf{Time discreteness} of operation of controlling system;
  \medskip  
  \item \textbf{Joint consideration} of blocks of controlling system;
  \medskip
  \item \textbf{Non-local} description of controlling system.
  \end{enumerate}    
\end{frame}


\begin{frame} {Moments of observation choice}
  \begin{figure}[h]
    \centering
    \pgfimage[height=2.5cm]{timings}
    \caption{Moments of observation scale}
    \label{VK:fig:3}
  \end{figure}
\end{frame}








\begin{frame}{Random variables}
  \begin{itemize}
    \item $\tau_i \in {\mathbb R}_+$, $i=0$, $1$, \ldots --- server state changing moments;
    \item $\eta_{j,i} \in Z_+$ --- number of customers in input flow $\Pi_j$ arrived during $(\tau_i, \tau_{i+1}]$;
    \item $\xi_{j,i} \in Z_+$ --- number of customers in saturation flow $\Pi^{\mbox{\scriptsize{sat}}}_j$ arrived during $(\tau_i, \tau_{i+1}]$;
    \item $\varkappa_{j,i}$ --- number of customers in queue $O_j$ at the moment $\tau_i$;
  \item $\Gamma_i\in\Gamma$ --- server state at the moment  $\tau_i$;
  \item $\overline{\xi}_{j,i} \in Z_+$ --- number of customers in output flow $\Pi^{\mbox{\scriptsize{out}}}_j$ serviced during $(\tau_i, \tau_{i+1}]$.
  \end{itemize}
  $j=1$,  $2$, $3$, $4$.
\end{frame}


\begin{frame}[allowframebreaks]{Transition graph}
 Server's state switching law: $\Gamma_{i+1}=h(\Gamma_i,\varkappa_{3,i})$.
   \begin{figure}[h]
    \centering
    \pgfimage[height=4.5cm]{SystemStates}
    \caption{Example. Lights concrete times}
    \label{VK:fig:3}
  \end{figure}
   \begin{figure}[h]
    \centering
    \pgfimage[height=6cm]{GraphScheme3}
    \caption{Example. Transition graph (mapping $h(\cdot,\cdot)$).}
    \label{VK:fig:3}
  \end{figure}

  \begin{equation}
  \Gamma = \bigl( \bigcup_{k=1}^d C_k \bigr) \bigcup \{\Gamma^{(0,1)}, \Gamma^{(0,2)}, \ldots, \Gamma^{(0,n_0)}\}, \quad C_k = C_k^{\mathrm{I}} \cup C_k^{\mathrm{O}}  \cup C_k^{\mathrm{N}}.
  \end{equation}
  \begin{equation}
h(\Gamma^{(k,r)},y) = 
\begin{cases}
\Gamma^{(k,r\oplus_k 1)},& \quad \text{ if } \Gamma^{(k,r)}\in C_k\setminus C_k^{\mathrm{O}};\\
\Gamma^{(k,r\oplus_k 1)},& \quad \text{ if } \Gamma^{(k,r)}\in C_k^{\mathrm{O}} \text{ и } y>L;\\
\Gamma^{(0,h_1(\Gamma^{(k,r)}))},& \quad \text{ if } \Gamma^{(k,r)}\in C_k^{\mathrm{O}} \text{ и } y\leqslant L;\\
\Gamma^{(0,h_2(r))},& \quad \text{ if } k=0 \text{ и } y\leqslant L;\\
h_3(r),& \quad \text{ if } k=0 \text{ and } y > L,
\end{cases}
\end{equation}
where 
$$h_1(\cdot)\colon \bigcup_{k=1}^d C_k^{\mathrm{O}}\to N_0, \quad h_2(\cdot)\colon N_0\to N_0, \quad h_3(\cdot)\colon N_0 \to\bigcup_{k=1}^d C_k^{\mathrm{I}},$$ и $N_0=\{1,2, \ldots, n_0\}$.
Then 
$\Gamma_{i+1} = h(\Gamma_i, \varkappa_{3,i}).
$
\end{frame}



   \begin{figure}[h]
    \centering
    \pgfimage[height=7cm]{GraphScheme3}
    \caption{Example. Transition graph}
    \label{VK:fig:4}
  \end{figure}




\begin{frame}{Functional relations}
System operation is subject to the following functional relationships:
\begin{align}
\overline{\xi}_{j,i}&=\min\{\varkappa_{j,i}+\eta_{j,i},\xi_{j,i}\}, \quad & j\in \{1,2,3\},\\
\varkappa_{j,i+1}&=\varkappa_{j,i}+\eta_{j,i}-\overline{\xi}_{j,i}, \quad & j\in \{1,2,3\},\\
\varkappa_{j,i+1}&=\max\{{0,\varkappa_{j,i}+\eta_{j,i}-\xi_{j,i}}\}, \quad & j\in \{1,2,3\},\\
\varkappa_{4,i+1}&=\varkappa_{4,i}+\eta_{4,i}-\eta_{2,i}, \quad &\\
\xi_{4,i} & = \varkappa_{4,i}, & \\
\eta_{4,i} & = \min\{ \varkappa_{1,i} + \eta_{1,i}, \xi_{1,i}\}.
\end{align}
\end{frame}





\begin{frame}{Low-priority queue sequence}
Low-priority queue is described with sequence
\begin{equation}
\label{eq:theMC:1}
\{(\Gamma_i(\omega), \varkappa_{3,i}(\omega)); i =0, 1, \ldots\}.
\end{equation}
\vfill
\begin{block}{\bf Theorem~1.}
Let $\Gamma_0=\Gamma^{(k,r)}\in \Gamma$ and $\varkappa_{3,0}=x_{3,0}\in \mathbb{Z}_+$ be fixed. Then the sequence~\eqref{eq:theMC:1} is Markov chain.
\end{block}

\begin{block}{\bf Theorem~3.}
For Markov chain~\eqref{eq:theMC:1} to have a stationary distribution it is sufficient to satisfy the following inequality
\begin{equation*}
\min_{k=\overline{1,d}} { \frac{\sum_{r = 1}^{n_k} \ell(k,r,3) }{\lambda_3 f_3'(1) \sum_{r=1}^{n_k} T^{(k,r)} }}>1.
\label{sufficient:double}
\end{equation*}
\end{block}

\end{frame}

\begin{frame}{Primary input flows queues sequence}
Primary input flows queues are described with sequence
\begin{equation}
\label{eq:theMC:2}
\{(\Gamma_i(\omega), \varkappa_{1,i}(\omega),\varkappa_{3,i}(\omega)); i =0, 1, \ldots\}.
\end{equation}
\vfill
\begin{block}{\bf Theorem~2.}
 Let $\Gamma_0=\Gamma^{(k,r)}\in \Gamma$ and $(\varkappa_{1,0}, \varkappa_{3,0})=(x_{1,0}, x_{3,0})\in \mathbb{Z}_+^2$ be fixed. Then the sequence~\eqref{eq:theMC:2} is Markov chain.
\end{block}
\begin{block}{\bf Theorem~4.}
For Markov chain~\eqref{eq:theMC:2} to have a stationary distribution it is sufficient to satisfy the following inequalities
\begin{equation*}
\min_{k=\overline{0,d}} { \frac{\sum_{r = 1}^{n_k} \ell(k,r,1) }{\lambda_1 f_1'(1) \sum_{r=1}^{n_k} T^{(k,r)} }}>1, \quad 
\min_{k=\overline{1,d}} { \frac{\sum_{r = 1}^{n_k} \ell(k,r,3) }{\lambda_3 f_3'(1) \sum_{r=1}^{n_k} T^{(k,r)} }}>1.
\label{sufficient:double}
\end{equation*}
\end{block}
\end{frame}


\begin{frame}{Intermediate queue sequence}
Intermediate queue is described with sequence
\begin{equation}
\label{eq:theMC:3}
\{(\Gamma_i(\omega), \varkappa_{1,i}(\omega),\varkappa_{3,i}(\omega),\varkappa_{4,i}(\omega)); i =0, 1, \ldots\}.
\end{equation}
\vfill
\begin{block}{\bf Theorem~5.}
For queue sizes in stochastic sequence~\eqref{eq:theMC:3} to be bounded it is sufficient to satisfy assumptions of \textit{Theorem~4} and following inequality:
\begin{equation*}
  % \min_{\substack{k=\overline{1,d}\\ j=1,3}} {\{p_{k,r}\}} > 0.
    \min_{k=\overline{0,d}, r=\overline{1,n_k}} {\{p_{k,r}\}} > 0.
\end{equation*}
\end{block}


\end{frame}





\begin{frame}
\Huge{\centerline{\color{blue} Thank you for your attention!}}
\end{frame}

\appendix
\section{Приложение}
\backupbegin

\begin{frame}[allowframebreaks]{Потоки первичных требований. Обозначения}
Пусть $\Gamma^{(k,r)}\in \Gamma$ и $x_3 \in Z_+$. Обозначим 
$$
{\mathbb H}_{-1}(\Gamma^{(k,r)}, x_3) = \{\gamma \in \Gamma \colon h(\gamma, x_3) = \Gamma^{(k,r)}\}.
$$
Вид отображения $h(\cdot,\cdot)$ позволяет записать явный вид множества ${\mathbb H}_{-1}(\Gamma^{(k,r)}, x_3)$:
\begin{equation*}
H_{-1}(\Gamma^{(k,r)}, x_3) = 
\begin{cases}
\bigl\{\Gamma^{(k_1,r_1)}, \Gamma^{(0,r\ominus_0 1)}\bigr\},&  \text{ if  $(k=0) \wedge (x_3 \leqslant L)$,}\\
\bigl\{\Gamma^{(k,r\ominus_k 1)}, \Gamma^{(0,r_2)}\bigr\},&  \text{ if  $(\Gamma^{(k,r)}\in C_k^{\mathrm{I}})
  \wedge (x_3>L)$,}\\ 
\bigl\{\Gamma^{(k,r\ominus_k 1)}\bigr\},&  \text{ if  $(\Gamma^{(k,r)}\in C_k^{\mathrm{O}}) \vee (\Gamma^{(k,r)}\in C_k^{\mathrm{N}})$;}\\
\varnothing,&  \text{ if  $(k = 0)\wedge  (x_3>L)$}\\
 & \qquad \text{ или $(\Gamma^{(k,r)}\in C_k^{\mathrm{I}}) \wedge (x_3\leqslant L)$}
\end{cases}
\end{equation*}
где $h_1(\Gamma^{(k_1,r_1)})=r$ и $h_3(r_2)=\Gamma^{(k,r)}$.
\framebreak

\end{frame}

\begin{frame}{Кодирование информации}
Пусть $Z_+$ --- множество целых неотрицательных чисел
  \begin{itemize}
  \item $\{e^{(1)}\}$ --- множество состояний \textbf{внешней среды} (одно состояние);
  \item $Z^4_+$ --- множество состояний \textbf{входных полюсов};
  \item $Z^4_+$ --- множество состояний \textbf{выходных полюсов};
 \item $\Gamma=\{\Gamma^{(k,r)} \colon k=0,1,\ldots,d; r=1,2,\ldots n_k\}$ --- множество состояний \textbf{внутренней памяти};
   \item $Z^4_+$ --- множество состояний \textbf{внешней памяти};
   \item $\{r^{(1)}\}$ --- множество состояний \textbf{устройства по переработке информации во внешней памяти} (одно состояние)
   \item граф переходов (будет описан ниже) описывает устройство по переработке информации во внутренней памяти
   \end{itemize}
\end{frame}

\begin{frame}[allowframebreaks]{Свойства условных распределений}
Определим функции $\varphi_j(\cdot,\cdot)$, $j\in \{1,3\}$, и $\psi(\cdot, \cdot, \cdot)$ из разложений:
\begin{equation*}
\sum_{\nu=0}^{\infty} z^\nu\varphi_j(\nu,t) = \exp\{\lambda_j t (f_j(z)-1)\}, \quad \psi(k;y,u)=C_y^k u^k (1-u)^{y-k}.	
\end{equation*}

Пусть $a=(a_1, a_2, a_3, a_4) \in \mathbb{Z}_+^4$ и $x=(x_1, x_2, x_3, x_4) \in \mathbb{Z}_+^4$.
%\begin{block}{Предположение 1}

Тогда вероятность $\varphi(a,k,r,x)$ одновременного выполнения равенств $\eta_{1,i}=a_1$, $\eta_{2,i}=a_2$, $\eta_{3,i}=a_3$, $\eta_{4,i}=a_4$ при условии  $\nu_i=(\Gamma{(k,r)}; x)$ есть 
\begin{equation}
\!\!\varphi_1(a_1,h_T(\Gamma^{({k},{r})},x_3)) \times \psi(a_2,x_4, p_{\tilde{k},\tilde{r}}) \times \varphi_3(a_3,h_T(\Gamma^{({k},{r})},x_3))
\times \delta_{a_4,\min{\{\ell(\tilde{k},\tilde{r},1), x_1+a_1}\}},
\end{equation}
%\end{block}
где
\begin{equation*}
\Gamma^{(\tilde{k},\tilde{r})}=h(\Gamma^{(k,r)},x_3), \quad \delta_{i,j}=\begin{cases} 1, \quad \text{ if }i=j\\0, \quad \text{ if } i\neq j,
\end{cases}.
\end{equation*}
и 
$$
T_{i+1}=h_T(\Gamma_i,\varkappa_{3,i})= T^{(k,r)},\quad  \Gamma^{(k,r)}=\Gamma_{i+1}=h(\Gamma_i,\varkappa_{3,i}).
$$
\framebreak

Пусть $b=(b_1, b_2, b_3, b_4) \in \mathbb{Z}_+^4$. 

Тогда вероятность $\zeta(b, k, r, x)$ одновременного выполнения равенств $\xi_{1,i}=b_1$, $\xi_{2,i}=b_2$, $\xi_{3,i}=b_3$, $\xi_{4,i}=b_4$ при фиксированном значении метки $\nu_i=(\Gamma{(k,r)}; x)$ есть
\begin{equation}
\delta_{b_1,\ell(\tilde{k},\tilde{r},1)} \times \delta_{b_2,\ell(\tilde{k},\tilde{r},2)} \times 
\delta_{b_3,\ell(\tilde{k},\tilde{r},3)} \times \delta_{b_4,x_4}.
\end{equation}
где $\tilde{k}$ и $\tilde{r}$ такие, что $\Gamma^{(\tilde{k},\tilde{r})}=h(\Gamma^{(k,r)},x_3)$.
\end{frame}
\begin{frame}[allowframebreaks]{Полученные ранее результаты}

\begin{block}
    {\bf Теорема 1.} {\it 
    Пусть $$\Gamma_0=\Gamma^{(k_0,r_0)} \in \Gamma,  \varkappa_{3,0}=x_3\in \mathbb{Z}_+$$ фиксированы. 
    
    Тогда последовательность $$\{(\Gamma_i, \varkappa_{3,i}); i \geqslant 0\}$$ является счетной цепью Маркова.}
\end{block}
\framebreak
\begin{block}
    {\bf Теорема 2.}{\it Пусть 
    $$x_3, \tilde{x}_3~\in \mathbb{Z}_+, \quad \Gamma^{(k,r)}\in \Gamma, \Gamma^{(\tilde{k},\tilde{r})}= h(\Gamma^{(k,r)},x_3).$$ Тогда условная вероятность 
$${\mathbf P}(\{\Gamma_{i+1}=\Gamma^{(\tilde{k},\tilde{r})},\varkappa_{3,i+1}=\tilde{x}_3\}|\{\Gamma_{i}=\Gamma^{(k,r)},\varkappa_{3,i}=x_3\})$$ равна
$$\delta_{\tilde{x}_3,0}\!\sum_{a=0}^{\ell(\tilde{k},\tilde{r},3)-x} \varphi_3(a,h_T(\Gamma^{(k,r)},x_3)) + (1-\delta_{\tilde{x}_3,0}) \varphi_3(\tilde{x}_3 + \ell(\tilde{k},\tilde{r},3)-x_3,h_T(\Gamma^{(k,r)},x_3)).$$}
\end{block}
\framebreak
\begin{block}
    {\bf Теорема 3.}{\it Пусть для $r=\overline{1,n_0}$ определено множество $$S^3_{0,r} =\bigl\{(\Gamma^{(0,r)},x_3)  \colon x_3\in Z_+,  L \geqslant x_3 > L - \max\bigl\{\sum_{t=0}^{n_k} \ell_{k,t,3}\colon k=\overline{1,d}\bigr\}\,\bigr\}
$$ и  для $k=\overline{1,d}$, $r=\overline{1,n_k}$ обозначено $$S^3_{k,r} =  \{(\Gamma^{(k,r)},x_3) \colon x_3\in Z_+, x_3 > L - \sum_{t=0}^{r-1} \ell_{k,t,3}\}.$$ Тогда множество существенных состояний марковской цепи $\{(\Gamma_i, \varkappa_{3,i}); i \geqslant 0\}$ есть $$S=\bigcup_{k=0}^d \bigl(\bigcup_{r=1}^{n_k} S^3_{k,r}\bigr).$$}
\end{block}

\end{frame}
\backupend\end{document}

%%%%%%%%%%%%%%%%%%%%%%%%%%%%%%%%%%%%%%%%%%%%%%%%%%%%%%%%%%%%%%%%%%%%%%%%%%%%%%%%
%%  Sample document for preparing papers for  "Upravlenie bolshimi sistemami"
%%  charset=windows-1251
%%  The sample is based on the analogous sample of "Avtomatika i Telemekhanika".
%%%%%%%%%%%%%%%%%%%%%%%%%%%%%%%%%%%%%%%%%%%%%%%%%%%%%%%%%%%%%%%%%%%%%%%%%%%%%%%%

\documentclass[11pt]{ubs}
%%
%% для просмотра "as is" необходим набор русских шрифтов А. Лебедева
%% http://tex.imm.uran.ru/texserver/fonts/pscyr/pscyr4c/
%%
\usepackage{graphicx}

\issue{XX}% Выпуск Сборника - проставляется в процессе верстки
\rubric{Рубрика Сборника}%  Рубрика Сборника - может быть скорректирована редактором
%
% Имеющиеся рубрики (последнюю информацию см. на сайте ubs.mtas.ru):
%
%    Системный анализ
%    Математическая теория управления
%    Анализ и синтез систем управления
%    Информационные технологии в управлении
%    Управление в социально-экономических системах
%    Управление в медико-биологических и экологических системах
%    Управление техническими системами и технологическими процессами
%    Управление подвижными объектами и навигация
%    Технические и программные средства управления
%    Надежность и диагностика средств и систем управления
%    Программы и системы моделирования объектов, средств и систем управления

% В соответствии с требованиями ВАК коды УДК и ББК обязательны. Проставляются автором. Могут быть скорректированы редактором
\udk{УДК ...} % Коды УДК можно найти на сайте https://classinform.ru/udk.html
\bbk{ББК ...} % Коды ББК можно найти на сайте https://classinform.ru/bbk.html

\title{Требования к оформлению статей в~сборник <<Управление большими системами>>}% Название статьи
\engtitle{Article Title} % В соответствии с требованиями ВАК обязательно указывается название статьи на английском языке.

% Ссылка на грант и прочие благодарности. Оставьте команду \footnote и просто измените ее аргумент.
% Если ссылка не нужна, просто удалите следующую строку.
\thanks{\footnote{Работа выполнена при финансовой поддержке РФФИ, грант №\ldots \\
Авторы признательны А.А.~Егорову за ценное обсуждение содержания
статьи.}}

% Список авторов должен вводиться в представленной форме. Авторы вводятся в алфавитном порядке.
% Сначала указывается Фамилия И.О., затем следует подстрочная ссылка (footnote) с полным именем,
% ученой степенью, должностью и контактной информацией (телефоном или e-mail-ом) автора,
% затем, с новой строки курсивом - организация, в которой работает автор.
% Если несколько подряд идущих авторов работают в одной организации, допускается перечислять их в одной строке,
% после которой идет строка с названием организации (см. пример ниже).
%
\authors{
%
\textbf{Иванов~А.А.}\footnote{Александр Алексеевич Иванов, кандидат
технических наук, доцент \emph{(}aaivanov@mail.ru\emph{)}.},
\textbf{Петров~Б.Б.}\footnote{Борис Борисович Петров, доктор физико-математических наук, профессор \emph{(}Москва, ул. Профсоюзная, д. 65, тел. \emph{(}495\emph{)} 000-00-00\emph{)}.}\\
\textit{\emph{(}Институт проблем управления им. В.А. Трапезникова РАН, Москва\emph{)}}\\
\textbf{Сидоров~В.В.}\footnote{Виктор Владимирович Сидоров, студент \emph{(}viktor.sidorov@mipt.ru\emph{)}.}\\
\textit{\emph{(}Московский физико-технический институт, Москва\emph{)}}
%
}
%
\engauthors{% % В соответствии с требованиями ВАК обязательно указывается информация об авторах на английском языке.
\textbf{Alexander Ivanov}, Institute of Control Sciences of RAS,
Moscow, Cand.Sc., assistant professor (aaivanov@mail.ru). \\
\textbf{Boris Petrov}, Institute of Control Sciences of RAS, Moscow,
Doctor of Science, professor (Moscow, Profsoyuznaya st., 65,
(495)000-00-00). \\
\textbf{Vikor Sidorov}, Moscow Institute of Physics and Technology,
Moscow, student (viktor.sidorov@mipt.ru). }

%
%
% Аннотация статьи.
% Обычно набирается в один абзац.
\abstract{%
Статья ОБЯЗАТЕЛЬНО начинается с аннотации. Аннотация должна быть независимым источником информации о статье. Она должна кратко и информативно раскрывать актуальность задачи, подход, методы и полученные результаты, позволять провести предварительную экспертизу и классификацию статьи. В аннотации недопустимо дословное повторение фрагментов текста статьи. Объем аннотации на русском языке – не~более 100~слов. Нежелательно включать в аннотацию формулы, таблицы, спецсимволы и ссылки на литературу.
}

% В соответствии с требованиями ВАК обязательно приводится расширенная аннотация на английском языке.
% Объем расширенной английской аннотации -- 150-200 слов.
% Английский текст проверяется редактором на предмет грамотности и соответствия русскому тексту и может быть изменен без согласования с авторами статьи.
\engabstract{%
150--200 words. Describes the standards of articles' formatting for "Large systems
control" papers collection. Provides the examples for typical
elements of an article. }

% Обязательно указывается от трёх до шести ключевых слов
\keywords{управление большими системами, электронное научное
издание, пример оформления статьи}

% В соответствии с требованиями ВАК обязательно приводится перевод ключевых слов на английский язык.
\engkeywords{large systems control, electronic scientific
publication, article formatting template}

\presented{...}% Член редколлегии, представляющий статью к публикации, заполняется редактором
\received{...}% Дата получения материала редакционной коллегией, заполняется  редактором
\published{...}% Дата опубликования статьи, заполняется редактором

\begin{document}

\maketitle

\section{Введение}

Если Вы не скачали только что этот файл с адреса http://ubs.mtas.ru/authors/rules.php, просим Вас скачать последнюю версию файла по этой ссылке. Приведенные в настоящем документе требования определяют формат предоставления материалов для опубликования в сборнике «Управление большими системами». Несоответствие присланных материалов описанным в данном файле требованиям может быть основанием отклонения статьи редактором сборника.

\textbf{Уважаемые авторы!} Поскольку это очень распространенная ошибка оформления, скажем сразу и повторим несколько раз. \textbf{Ссылки на литературу нумеруются по алфавиту, по фамилии первого автора}, а не по порядку упоминания их в тексте. Ваша статья не будет опубликована, пока это требование не будет в точности соблюдено.

\textbf{Внимание!} В процессе подготовки статьи к публикации редактор может попросить Вас доработать как оформление (форматирование), так и текст статьи. Пожалуйста, будьте готовы оперативно учитывать замечания редактора и следуйте его указаниям.

В следующих разделах приводятся примеры оформления различных элементов статьи и текста. Настоятельно рекомендуем применять готовые окружения из данного шаблона (sample.tex). Не подключайте дополнительные и экзотические пакеты, если можно обойтись стандартным набором команд и символов.

\section{Постановка задачи}
\subsection{\MakeUppercase{Заголовок подраздела}}
Перечни кодов УДК и ББК можно найти в интернете:
\begin{itemize}
\item УДК – https://classinform.ru/udk.html;
\item ББК – https://classinform.ru/bbk.html.
\end{itemize}
\subsection{\MakeUppercase{Примеры оформления}}
\begin{example}
В данном примере используются формулы. В следующих разделах приводятся и другие
примеры формул, в том числе занимающих несколько строк.
\begin{equation}
    2\times 2=4.
\end{equation}
Все добавляемые примеры автоматически заканчиваются символом
<<$\bullet$>>. Проблема может возникнуть только если пример
оканчивается рисунком или формулой (вертикальным элементом). Тогда
символ <<$\bullet$>> будет размещен на следующей строчке, что не
очень красиво.
\end{example}

Введем следующее определение.

\begin{definition}
Хххххххххххххх хххххх-хххххххх х ххххх ххххххххх ххх ххххххх хххх.
Ххххх хххххххх хххххххххххххх х ххххх.
\end{definition}

Рассмотрим следующую задачу.

\begin{problem} \label{prob:1}
Хххххх хххххххх хххххх хххххххх х ххххх ххххххххх ххх ххххххх хххх.
Ххххх хххххххх хххххххххххххх х ххххххххх х ххххх.
\end{problem}

Хххххх хххххххх хххххх хххххххх х ххххх ххххххххх ххх ххххххх хххх.
Ххххх хххххххх хххххххххххххх х ххххххххх х ххххх.


\section{Решение задачи}
%Заголовки подразделов делаются заглавными буквами
\subsection{\MakeUppercase{Заголовок подраздела делаются заглавными буквами}}

В соответствии с российскими стандартами дробную часть числа отделяйте от целой части запятой, а не точкой:
\begin{equation}
1{,}5^2=2{,}25. \nonumber
\end{equation}

Приведем пример оформления теоремы.\footnote{Это пример сноски.}

\begin{theorem}[{\cite[c.\,123]{first}}] %
Пусть выполнены следующие условия:

\begin{ruslist}
\item
первое условие;

\item
второе условие.
\end{ruslist}
Тогда справедливы следующие утверждения:

\begin{enumlist}
\item
первое утверждение;

\item
второе утверждение.
\end{enumlist}
\end{theorem}

\begin{proof}
Хххххххххххххх хххххххххххххх х хххххххххххххх ххх ххххххх хххх.
Хххххххххххххх хххххххххххххх х хххххххххххххх ххх ххххххх хххх:
\begin{gather}
    \sum_{n=1}^{\infty} 2^{-n} = 1.
\end{gather}
Хххххххххххххх хххххххххххххх х хххххххххххххх ххх ххххххх хххх,
хххххххххххххх хххххххххххххх х хххххххххххххх ххх ххххххх хххх?
Хххххххххххххх хххххххххххххх х хххххххххххххх ххх ххххххх хххх!!!
\end{proof}
Хххххххххххххх хххххххххххххх х хххххххххххххх ххх ххххххх хххх.

\begin{corollary}
Хххххххххххххх хххххххххххххх х хххххххххххххх ххх ххххххх хххх.
Хххххххххххххх хххххххх х хххххххххххххх ххх ххххххх хххх.
\end{corollary}

Хххххххххххххх хххххххххххххх х хххххххххххххх ххх ххххххх хххх.


\subsection{\MakeUppercase{Заголовок следующего подраздела}}

Хххххххххххххх хххххххххххххх х хххххххххххххх ххх ххххххх хххх.
Хххххххххххххх хххххххххххххх х хххххххххххххх ххх ххххххх хххх.

\begin{lemma}[(см.\ {\cite[c.\,45]{second}})] \label{lm:1}
Хххххххххххххх хххххххххххххх х хххххххххх:
\begin{multline}
    2\times 2\times 2\times 2\times 2\times 2\times 2\times 2\times
\\
    2\times 2\times 2\times 2\times 2\times 2\times 2\times 2\times
\\
    \times 2\times 2\times 2\times 2\times 2\times 2\times 2\times 2\times
\\
    2\times 2\times 2\times 2\times 2\times 2\times 2\times 2
\end{multline}
хххххххх ххххххх ххххххх х хххххххххххххх ххх.
\end{lemma}

Опишем следующий алгоритм.

\begin{algorithm}[(Быстрый)] \label{alg:1}
\ %%<-- этот пробел для того, чтобы первый элемент перечня был
%% на следующей строке, а не в подбор к заголовку окружения

\begin{enumlist}[.] % перечни, нумеруемые 1. 2. и т.д.
% \setcounter{enumlisti}{-1} % <-- эта команда нужна
%% для нумерации элементов перечня с нулевого
\item
Начать.

\item
Изменить.

\item
Закончить.
\end{enumlist}
\end{algorithm}

Для набора таблиц используются стандартные средства TeX.

\begin{table}[htbp]
\caption{Такая вот таблица}%
\label{tb:1}
\begin{tabular}{|r|c|l|}
\hline
Вы можете   & использовать           & стандартные \\
средства    & \TeX \space для набора & различных   \\ \cline{2-2}
таблиц.     & При этом               & перед таблицей \\
\hline \hline
должен идти & заголовок              & с ее номером и названием. \\
\hline
\end{tabular}
\end{table}

В силу формата А5 рисунки предпочтительно вставлять в текст без
обтекания, по центру (см. рис.~\ref{fig:direct}).

\begin{figure}[htbp]
\begin{picture}(60,40)
\put(0,0){\vector(1,0){2}} \put(60,40){\vector(-1,0){2}}
\put(30,20){\vector(1,0){30}} \put(30,20){\vector(4,1){20}}
\put(30,20){\vector(3,1){25}} \put(30,20){\vector(2,1){30}}
\put(30,20){\vector(1,2){10}} \thicklines
\put(30,20){\vector(-4,1){30}} \put(30,20){\vector(-1,4){5}}
\thinlines \put(30,20){\vector(-1,-1){5}}
\put(30,20){\vector(-1,-4){5}}
\end{picture}
  \caption{ Такой вот рисунок \emph{(}по данным из таблицы \ref{tb:1}\emph{)}}
  \label{fig:direct}
\end{figure}

Следующая теорема утверждает сходимость алгоритма~\ref{alg:1}.

\begin{theorem}[(Теорема сходимости)] \label{th:2}
Алгоритм~{\rm\ref{alg:1}} сходится.
\end{theorem}

Можно привести замечание.

\begin{remark}
Ххххххххххх хххххххххххххх х хххххххххххххх ххх ххххххх хххх.
Ххххххххххх хххххххххххххх х хххххххххххххх ххх ххххххх хххх.
\end{remark}

\section{Примеры}

Данный раздел содержит несколько примеров различных окружений.

\begin{example}
Хххххххххххххх хххххххххххххх х хххххххххххххх ххх ххххххх хххх.
Хххххххххххххх хххххххххххххх х хххххххххххххх ххх ххххххх хххх.

Некоторые перечни можно нумеровать русскими буквами:
\begin{ruslist} % перечни, нумеруемые а), б) и т.д.
\item
ххххххххххххх хххххххххххххх х хххххххххххххх ххх ххххххх хх
ххххххххххххх хххххххххххххх х хххххххххххххх ххх ххххххх хххх;

\item
ххххххххххххх хххххххххххххх х хххххххххххххх ххх ххххххх хххх.
\end{ruslist}

А некоторые можно и латинскими буквами, хотя в русскоязычной статье этого лучше не делать :

\begin{latlist} % перечни, нумеруемые а), б) и т.д.
\item
ххххххххххх ххх ххххххх хххх;
\item
хххххххххххх ххх ххххххх хххх.
\end{latlist}
Ххххххххххх хххххххххххххх х хххххххххххххх ххх ххххххх хххх.
Ххххххххххх хххххххххххххх х хххххххххххххх ххх ххххххх хххх.
\end{example}

Можно сформулировать утверждение.

\begin{statement}
Ххххххххххх хххххххххххххх х хххххххххххххх ххх ххххххх хххх.
Ххххххххххх хххххххххххххх х хххххххххххххх ххх ххххххх хххх.
\end{statement}

\section{Результаты}

Можно сформулировать предложение.\penalty-10
\begin{proposition}
Ххххххххххх хххххххххххххх х хххххххххххххх ххх ххххххх хххх.
\end{proposition}

\section{Выводы и перспективы}
Все замечания и предложения по данному стилевому файлу и примерам оформления просим направлять по электронной почте ответственному секретарю редакционной коллегии сборника <<Управление большими системами>> (см. сайт ubs.mtas.ru).

\textbf{Уважаемые авторы!} Поскольку это очень распространенная ошибка оформления, подчеркнем еще раз.\textbf{ Ссылки на литературу нумеруются по алфавиту, по фамилии первого автора}, а не по порядку упоминания их в тексте. \textbf{Сначала идут кириллические публикации, потом латиница (как правило, англоязычные).} Ваша статья не будет опубликована, пока это требование не будет Вами в точности соблюдено. Строго соблюдайте правила оформления библиографических ссылок (см. примеры ниже). Просим Вас с пониманием отнестись к просьбе редактора об оперативном учете этих требований.




% Авторы - по алфавиту, сначала русскоязычные, затем остальные.
\begin{thebibliography}{10}

\bibitem{first}
{\MakeUppercase{Адорно~Т.В.}} \textit{К логике социальных наук}~// Вопросы философии.~-- 1992.~-- №10.~-- С.~76--86.

\bibitem{second}
{\MakeUppercase{Большов~П.П.}} \textit{Название книги}.~-- М.: Изд-во <<УБС>>, 2017.~-- 142~c.


\bibitem{r4}{\MakeUppercase{Глухов~В.А.}} \textit{Исследование, разработка и построение системы электронной доставки документов в библиотеке}: Автореф. дис. канд. техн. наук. – Новосибирск, 2000. – 18 с.

\bibitem{r5}{\MakeUppercase{Еськов~Д.Н., Бонштедт~Б.Э., Корешев~С.Н., Лебедева~Г.И., Серегин~А.Г.}} \textit{Оптико-электронный аппарат}~// Патент России №2122745.~-- 1998.~-- Бюл.~№33.

\bibitem{r6}{\MakeUppercase{Корнилов В.И.}} \textit{Турбулентный пограничный слой на теле вращения при периодическом вдуве/отсосе}~// Теплофизика и аэромеханика.~-- 2006.~-- Т.~13, №3.~-- С.~369--385.

\bibitem{r7}{\MakeUppercase{Кузнецов~А.Ю.}} \textit{Консорциум -- механизм организации подписки на электронные ресурсы}~// Российский фонд фундаментальных исследований: десять лет служения российской науке.~-- М.: Научный мир, 2003.~-- С.~340--342.

\bibitem{r8}{\MakeUppercase{Литчфорд~Е.У.}} \textit{С Белой Армией по Сибири} [Электронный ресурс]~// Восточный фронт Армии Генерала А.В.~Колчака: сайт.~-- URL: http://east-front.narod.ru/memo/latchford.htm (дата обращения: 23.08.2007).

\bibitem{r9}{\MakeUppercase{Логинова~Л.Г.}} \textit{Сущность результата дополнительного образования детей}~// Образование: исследовано в мире: междунар. науч. пед. интернет-журн.~-- 21.10.03.~-- URL: http://www.oim.ru/reader.asp?nomer=366 (дата обращения: 17.04.07).

\bibitem{r10}\textit{Официальные периодические издания : электронный путеводитель}~/ Рос. нац. б-ка, Центр правовой информации. [СПб.], 2005--2007.~-- URL:  http://www.nlr.ru/lawcenter/ izd/index.html (дата обращения: 18.01.2007).

\bibitem{r11}Патент РФ №2000130511/28, 04.12.2000.

\bibitem{r12}{\MakeUppercase{Райзберг~Б.А.,  Лозовский~Л.Ш., Стародубцева~Е.Б.}} \textit{Современный экономический словарь.} 5-е изд., перераб. и доп.~-- М.:ИНФРА-М, 2006.~-- 494~с.

\bibitem{r13}\textit{Рынок тренингов Новосибирска: своя игра} [Электронный ресурс].~-- Режим доступа: http://nsk.adme.ru/news/2006/07/ 03/2121.html.

\bibitem{r14}{\MakeUppercase{Тарасова~В.И.}} \textit{Политическая история Латинской Америки}: учеб. для вузов. 2-е изд.~-- М.: Проспект, 2006.~-- С.~305--412.

\bibitem{r15}{\MakeUppercase{Фенухин В. И.}} \textit{Этнополитические конфликты в современной России: на примере Северо-Кавказского региона}: дис. канд. полит. наук.~-- М., 2002.~-- С.~54--55.

\bibitem{r16}\textit{Философия культуры и философия науки: проблемы и гипотезы}: межвуз. сб. науч. тр. / Сарат. гос. ун-т; [под ред. С.Ф.~Мартыновича].~-- Саратов: Изд-во Сарат. ун-та, 1999.~-- 199~с.

\bibitem{r17}\textit{Экономика и политика России и государств ближнего зарубежья}: аналит. обзор, апр. 2007 / Рос. акад. наук, Ин-т мировой экономики и междунар. отношений.~-- М.: ИМЭМО, 2007.~-- 39~с.

\bibitem{r18}{\MakeUppercase{Crawford~P.J.,  Barrett~T.P.}} \textit{The reference librarian and the business professor: a strategic alliance that works}~// Ref. Libr.~-- 1997.~-- Vol.~3, No.~58.~-- P.~75--85.

\bibitem{r19}http://www.nlr.ru/index.html (дата обращения: 20.02.2007).

\end{thebibliography}

\makeenginfo
\makeauxinfo

\end{document}


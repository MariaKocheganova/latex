\documentclass[10pt]{beamer}
\usepackage[T2A]{fontenc}
\usepackage[russian]{babel}
\usepackage{amsmath}

\usepackage{beton}

\usepackage{tikz}
\usepackage[russian]{babel}
\usepackage[utf8]{inputenc}



\defbeamertemplate{footline}{centered page number}
{%
  \hspace*{\fill}%
  \usebeamercolor[fg]{page number in head/foot}%
  \usebeamerfont{page number in head/foot}%
  \insertpagenumber\,/\,\insertpresentationendpage%
  \hspace*{\fill}\vskip2pt%
}
\setbeamertemplate{footline}[centered page number]

\mode<presentation>{
%  \usetheme{Madrid}
  \usetheme{AnnArbor}
  \usefonttheme{serif}
}

%\setbeamertemplate{frametitle continuation}[from second]

\author[В.М. Кочеганов, А.В. Зорин]{\textbf{В.М. Кочеганов}, А.В. Зорин}
\title[Вероятностная модель тандема...]%
{ВЕРОЯТНОСТНАЯ МОДЕЛЬ ТАНДЕМА СИСТЕМ МАССОВОГО ОБСЛУЖИВАНИЯ С~ЦИКЛИЧЕСКИМ УПРАВЛЕНИЕМ С~ПРОДЛЕНИЕМ}
\institute[ННГУ]{\normalsize Нижегородский
  государственный университет \\  им. Н.И. Лобачевского}
\date[23-26.02.2015]{
Теория вероятностей, случайные процессы, \\
математическая статистика и приложения\\
(г. Минск, Республика Беларусь)}
\begin{document}

\begin{frame}
  \maketitle
\end{frame}

\begin{frame}{Физическая постановка задачи}
  \begin{figure}[h]
    \centering
    \pgfimage[height=5cm]{Crossroads}
    \caption{Тандем перекрёстков}
    \label{VK:fig:1}
  \end{figure}
\end{frame}

\begin{frame}{Общая постановка задачи}
  \begin{figure}[h]
    \centering
    \pgfimage[height=6cm]{SystemScheme}
    \caption{Структурная схема системы массового обслуживания}
    \label{VK:fig:2}
  \end{figure}
\end{frame}

\begin{frame}{Параметры системы}
\begin{block}{Входные потоки}
Входные потоки $\Pi_1$ и $\Pi_3$ --- \textbf{неординарные пуассоновские} потоки с интенсивностями $\lambda_1$ и $\lambda_3$ соответственно. 

Распределение числа заявок в группе по потоку $\Pi_j$, $j\in \{1,3\}$, имеет \textbf{производящую функцию}:
$$
f_j(z) = \sum_{\nu=1}^{\infty} p_{\nu}^{(j)} z ^{\nu}, \quad |z|<(1+\varepsilon), \varepsilon>0.
$$

Требования входных потоков $\Pi_4$ и $\Pi_2$ формируются из выходных требований потоков $\Pi^{\mbox{\scriptsize{вых}}}_1$ и $\Pi^{\mbox{\scriptsize{вых}}}_4$ соответственно.
\end{block}


\begin{block}{Дисциплины очередей}
Устройство $\delta_j$, $j \in \{1, 2, 3, 4\}$, поддерживает FIFO дисциплину очереди $O_j$.
\end{block}

\end{frame}


\begin{frame}{Классический подход}
\begin{block}{}
Представлен, например, в работах А.~Эрланга, А.Я.~Хинчина, А.Н.~Колмогорова, Д.~Кендалла, Б.В.~Гнеденко.
\end{block}

  Требуется задать:
  \begin{itemize}
  \item \textbf{входной поток} в виде конечномерных распределений процесса $\{\eta(t) \colon t \geqslant 0\}$, где $\eta(t)$ есть число поступивших требований до момента времени $t$;
  \item \textbf{процесс обслуживания} в виде интегральной функции
    распределения длительности обслуживани произвольной заявки;
  \item \textbf{дисциплину формирования очереди} в виде словесного описания на
    содержательном уровне;
  \item \textbf{структуру системы} в виде словесного описания на
    содержательном уровне.
  \end{itemize}
\end{frame}


\begin{frame}{Асимптотический подход}
\begin{block}{}
Сформулирован и развивается в работах А.А.~Боровкова.
\end{block}

  Требуется задание \textbf{случайного процесса} \[\{(\eta(t),
  \nu(t), \zeta(t));t\geqslant0\}, \] где 
  \begin{itemize}
  \item $\eta(t)$ --- число поступивших заявок за промежуток $[0,t)$;
  \item $\nu(t)$ --- число получивших отказ заявок за промежуток $[0,t)$;
  \item $\zeta(t)$ --- число обслуженных заявок за промежуток $[0,t)$;
  \end{itemize}
  Основная цель --- изучение общих предельных свойств распределения
  длины очереди \[\eta(t)-\nu(t)-\zeta(t). \]
\end{frame}

\begin{frame}{Основные недостатки перечисленных методов}
  \begin{itemize}
  \item Не удается решить проблему изучения \textbf{выходных потоков}.
    \bigskip
  \item Не удается рассмотреть \textbf{системы с немгновенным перемещением требований} между узлами и с зависимыми, разнораспределенными длительностями обслуживания требований.
  \end{itemize}
\end{frame}


\begin{frame}[allowframebreaks]{Кибернетический подход}
\begin{block}{}
Представлен, например, в следующих работах:
\begin{itemize}
\item  С.~В.~Яблонский. Основные понятия кибернетики~// Проблемы
  кибернетики. Вып. 2. М.: Физматгиз, 1959. С.~7--38.
\item М.~А.~Федоткин. Процессы обслуживания и
  управляющие системы // Математические проблемы
  кибернетики. Вып. 6. М.: Наука. 1996. С.~51--70.
\item А.~В.~Зорин, М.~А.~Федоткин. Оптимизация управления дважды
  стохастическими неординарными потоками в системах с разделением
  времени // Автоматика и телемеханика. \No~7, 2005. С.~102--111.
\end{itemize}
\end{block}

  \framebreak

Требуется задание:
\begin{itemize}
\item дискретных моментов наблюдения за системой с помощью точечного случайного процесса $\{\tau_i, i\geqslant0\}$;
\item случайных величин и случайных элементов, описывающих СМО и соответствующим моментам наблюдения $\tau_i$.
\end{itemize}

   \begin{figure}[h]
    \centering
    \pgfimage[height=2cm]{timings}
    \caption{Шкала моментов наблюдения}
    \label{AZ:fig:3}
  \end{figure}
\framebreak
Основные принципы кибернетического подхода:
  \begin{enumerate}
  \item \textbf{принцип дискретности} актов функционирования управляемой
    системы обслуживания во времени;
  \item \textbf{принцип нелокальности} при описании поэлементного
    строения управляемой системы обслуживания;
  \item \textbf{принцип совместного
    рассмотрения} поэлементного строения управляющей системы
    обслуживания и ее функционирования во времени.
  \end{enumerate}
\framebreak
Элементы управляющей кибернетической системы:
  \begin{itemize}
  \item \textbf{схема}
  \begin{itemize}
      \item внешняя среда
      \item входные и выходные полюса
      \item внешняя и внутренняя память
      \item устройства по переработке информации во входной и выходной памяти
      \end{itemize}
  \item \textbf{информация} --- набор состояний среды, очередей в накопителях, обслуживающего устройства, потоков насыщения и потоков обслуженных требований
  \item \textbf{координаты} --- номера состояний случайной среды, входных потоков, накопителей, механизмов по формированию очереди и номер состояния обслуживающего устройства
  \item \textbf{функция} --- обслуживание потоков по циклическому алгоритму
  \end{itemize}

\end{frame}

\begin{frame}{Кодирование информации}
  \begin{itemize}
    \item $\tau_i \in {\mathbb R}_+$, $i=0$, $1$, \ldots --- момент смены состояния
    обслуживающего утройства;
    \item $\eta_{j,i} \in Z_+$ --- число требований потока $\Pi_j$, поступивших за
    промежуток $(\tau_i, \tau_{i+1}]$;
    \item $\xi_{j,i} \in Z_+$ --- число требований потока насыщения $\Pi^{\mbox{\scriptsize{нас}}}_j$ на промежутке $(\tau_i, \tau_{i+1}]$;
    \item $\varkappa_{j,i}$ --- число требований в
    очереди $O_j$ в момент $\tau_i$;
  \item $\Gamma_i\in\Gamma=\{\Gamma^{(k,r)} \colon k=0,1,\ldots,d; r=1,2,\ldots n_k\}$ --- состояние обслуживающего устройства в момент $\tau_i$;
  \item $\overline{\xi}_{j,i} \in Z_+$ --- число требований
    выходного потока $\Pi^{\mbox{\scriptsize{вых}}}_j$ на промежутке
    $(\tau_i, \tau_{i+1}]$.
  \end{itemize}
  $j=1$,  $2$, $3$, $4$.
\end{frame}

\begin{frame}{Рекуррентные соотношения}
Функционирование системы подчиняется следующим функциональным соотношениям:
\begin{equation}
\begin{aligned}
\overline{\xi}_{j,i}&=\min\{\varkappa_{j,i}+\eta_{j,i},\xi_{j,i}\}, \quad & j\in \{1,2,3\},\\
\varkappa_{j,i+1}&=\varkappa_{j,i}+\eta_{j,i}-\overline{\xi}_{j,i}, \quad & j\in \{1,2,3\},\\
\varkappa_{j,i+1}&=\max\{{0,\varkappa_{j,i}+\eta_{j,i}-\xi_{j,i}}\}, \quad & j\in \{1,2,3\},\\
\varkappa_{4,i+1}&=\varkappa_{4,i}+\eta_{4,i}-\eta_{2,i}, \quad &\\
\xi_{4,i} & = \varkappa_{4,i}. &
\end{aligned}
\label{rek}
\end{equation}


\end{frame}

\begin{frame}{Реализуемость физических предположений}
  \begin{block}
    {\bf Теорема 1.} {\it 
    Пусть 
    $$\gamma_0=\Gamma^{(k_0,r_0)} \in \Gamma, \quad x^0=(x_{1,0},x_{2,0}, x_{3,0},x_{4,0})\in \mathbb{Z}_+^4$$ фиксированы. Тогда существует вероятностное пространство $(\Omega, {\cal F}, {\mathbf P}(\cdot))$ и заданные на нем случайные величины $\eta_{j,i}=\eta_{j,i}(\omega), \xi_{j,i}=\xi_{j,i}(\omega), \varkappa_{j,i}=\varkappa_{j,i}(\omega)$ и случайные элементы $\Gamma_i=\Gamma_i(\omega)$, $i\geqslant 0$, $j\in \{1, 2, 3, 4\}$, такие, что 
    \begin{itemize}
    \item имеют место равенства $\Gamma_0(\omega) = \gamma_0$ и $\varkappa_0(\omega)=x^0$;
    \item выполняются соотношения \eqref{rek};
    \item для любых  $a$, $b$, $x^t=(x_{1,t},x_{2,t},x_{3,t},x_{4,t}) \in \mathbb{Z}_+^4$, $\Gamma^{(k_t,r_t)} \in \Gamma$, $t = 1, 2, \ldots$, условное распределение векторов $\eta_i$, и $\xi_i$ имеет вид 
\begin{multline*}
    {\mathbf P}(\{ \omega \colon \eta_i = a, \xi_i=b\} |\bigcap_{t=0}^{i}\{\omega\colon \Gamma_t=\Gamma{(k_t,r_t)}, \varkappa_t=x^t\})=\\
=\varphi(a,k_i,r_i,x^i)\times \zeta(b,k_i,r_i,x^i),
\end{multline*}
    \end{itemize}
}
  \end{block}
\end{frame}

\begin{frame}{Марковость последовательности $\{(\Gamma_i, \varkappa_i); i \geqslant 0\}$}
\begin{block}
    {\bf Теорема 2.} {\it 
Пусть $\Gamma_0=\Gamma^{(k,r)}\in \Gamma$ и $\varkappa_0=x^0\in \mathbb{Z}_+^4$ фиксированы. Тогда последовательность $\{(\Gamma_i, \varkappa_i); i \geqslant 0\}$ является однородной счетной цепью Маркова.
}
\end{block}
\end{frame}
\end{document}

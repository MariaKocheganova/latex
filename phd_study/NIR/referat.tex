\documentclass[14pt]{extarticle}

\usepackage[margin=2cm]{geometry}

\usepackage[T2A]{fontenc}
\usepackage[utf8]{inputenc}
\usepackage[russian]{babel}
\usepackage[onehalfspacing]{setspace}
\usepackage{amsthm,amsmath}
\usepackage{amssymb,amsfonts}

\usepackage{indentfirst}

\setlength{\parindent}{1.25cm}

\newtheorem{definition}{Определение}
\newtheorem{theorem}{Теорема}

\renewcommand{\Pr}{\mathbb P}


\makeatletter
\renewcommand{\@biblabel}[1]{#1.}
\renewenvironment{thebibliography}[1]
     {\section*{\refname}%
       \addcontentsline{toc}{section}{\refname}%
       \addcontentsline{tce}{section}{References}%
      \@mkboth{\MakeUppercase\refname}{\MakeUppercase\refname}%
      \list{\@biblabel{\@arabic\c@enumiv}}%
           {\settowidth\labelwidth{\@biblabel{#1}}%
            \leftmargin 0pt % \labelwidth 
            \itemindent \parindent
            \advance\itemindent\labelwidth
            % \advance\leftmargin\labelsep
            \itemsep 0ex \parsep 0ex
            \@openbib@code
            \usecounter{enumiv}%
            \let\p@enumiv\@empty
            \renewcommand\theenumiv{\@arabic\c@enumiv}}%
      \sloppy
      \clubpenalty4000
      \@clubpenalty \clubpenalty
      \widowpenalty4000%
      \sfcode`\.\@m}
     {\def\@noitemerr
       {\@latex@warning{Empty `thebibliography' environment}}%
      \endlist}
\makeatother



\begin{document}
\begin{titlepage}
  \begin{center}
    МИНИСТЕРСТВО ОБРАЗОВАНИЯ И НАУКИ \\РОССИЙСКОЙ ФЕДЕРАЦИИ \\
    \textbf{%
      Федеральное государственное автономное \\
      образовательное учреждение высшего образования  \\
      <<Национальный исследовательский
      Нижегородский государственный университет им. Н.И. Лобачевского>> \\
      (ННГУ)
    }

 \medskip
 
    Институт информационных технологий, математики и механики \\
    Кафедра программной инженерии

    \medskip

    Направление подготовки: <<Математика и механика>>\\
    Направленность подготовки: <<Теория вероятностей и математическая
    статистика>>

    \vfill

    РЕФЕРАТ \\ по дисциплине <<Введение в общие цепи Маркова>>\\
    по теме: <<Эргодические цепи Маркова>>

    \vfill \hfill
    \begin{minipage}[h]{ 0.5\linewidth}
      ВЫПОЛНИЛ: \\
      аспирант кафедры МОСТ\\
      третьего года обучения\\
      Кочеганов В.М.
    \end{minipage}
    \vfill {Н.Новгород, 2016}
  \end{center}
\end{titlepage}

\tableofcontents

\newpage

\section{Понятие общей цепи Маркова}


Пусть задано вероятностное пространство $(\Omega, \mathfrak F, \Pr)$, где
$\Omega$ --- множество описаний $\omega$ элементарных исходов, $\mathfrak F$~---
сигма-алгебра событий $A\subset \Omega$, $\Pr$~--- вероятностная мера на
$\mathfrak F$. 

\begin{definition}{\cite{Shyr}}
  Последовательность случайных элементов $\{X_n(\omega); n=0, 1, \ldots\}$ со
  значениями в измеримом пространстве $(E, \mathcal E)$ образует общую цепь
  Маркова, если для любого множества $B\in\mathcal E$ 
  \[
  \Pr(X_{n+1}\in B \mid X_0, \ldots, X_n) = \Pr(X_{n+1}\in B \mid X_n) \quad
  \text{$\Pr$-почти наверное}.
  \]
  Цепь Маркова называется однородой, если вероятность справа может быть выбрана
  так, чтобы не зависеть от индекса $n$.
\end{definition}


Для общей цепи Маркова важную роль играет ее переходное ядро --- вариант
условной вероятности 
\[
P(x,B) = \Pr(X_{n+1}\in B \mid X_n=x).
\]
По заданному переходному ядру $P(\cdot, \cdot)$ и произвольному распределению
$\nu(\cdot)$ вероятностей на измеримом пространстве $(E, \mathfrak E)$ можно
определить \emph{выборочное} вероятностное пространство $(E^\infty, \mathcal
B(E^\infty), P_X)$ и определить <<координатную>> цепь Маркова $X=\{X_n(\omega);
n=0, 1, \ldots\}$ с $X_n(\omega)=\omega_n$ для $\omega=(\omega_0, \omega_1,
\ldots)\in E^\infty$, такую что для любых $B_0$, $B_1$, \ldots, $B_n\in \mathcal
E^{n+1}$ выполняется равенство
\[
P_X(X_0\in B_0, X_1\in B_1, \ldots, X_n\in B_n)=\int_{B_0} \nu(d\omega_0)
\int_{B_1} P(\omega_0, d\omega_1) \cdots \int_{B_n} P(\omega_{n-1}, d\omega_n).
\]

Классическая цепь Маркова с дискретным временем и счетным множеством состояний~\cite{Kemeni:Snell}
являются частным случаем общей цепи Маркова. Если обозначить состояния числами
$0$, $1$, \ldots, а вероятность перехода за один шаг из состояния $i$ в $j$ как
$p_{i,j}$, то $E=\{0, 1, \ldots\}$, $P(x,B)=\sum_{j\in B} p_{i,j}$. 

Примером общей цепи Маркова с непрерывным пространством состояний является
векторный процесс авторегрессии~\cite{Hamilton} вида $X_{n+1}=\Phi X_n+\varepsilon_{n}$, где $X_n\in
R^m$, $n=0$, $1$, \ldots{} и $\varepsilon_n$, $n=0$, $1$, \ldots{}~--- случайные
$m$-мерные векторы, $\epsilon_n \sim \mathcal N(\mathbf 0, \Sigma)$. Здесь
$\Sigma$~--- некоторая положительно определенная матрица (ковариационная матрица
вектора $\varepsilon_n$). В этом случае переходная вероятность $P(x,B)$ имеет
$m$-мерную плотность $p(x,y)$ относительно меры Лебега:
\[
p(x,y)=(2\pi)^{-m/2}(\det \Sigma)^{-1/2} \exp\Bigl\{-\frac12 (x-y) \Sigma^{-1}
(x-y)^T\Bigr\}.
\]



Для выяснения структуры фазового пространства общей цепи Маркова вводится
понятие циклов.

\begin{definition}
  Последовательность $E_0$, $E_1$, \ldots, $E_{m-1}$ непересекающихся
  подмножеств из $\mathcal E$
  называется {\emph{$m$-циклом}}, если для
  каждого $i=0$, $1$, \ldots, $m-1$ и для каждого $x\in E_i$ оказывается
  \[
  P(x,E_j)=1, \qquad j\equiv i+1 \mod m.
  \]
\end{definition}


Обобщением понятия сообщающихся состояний на произвольное фазовое пространство
является понятие неприводимости. Введем функцию
\[
L(x,A)=\Pr\Bigl(\mathop{\cup}_{n=1}^\infty \{X_n\in A\}\Big|\,\{X_0=x\}\Bigr)
\]

\begin{definition}\label{AZ:def:irreducible} \sloppy
  Стохастическое ядро $P(\cdot,\cdot)$ называется
  {\emph{$\varphi$-не\-при\-во\-димым}}, если существует мера $\varphi(\cdot)$
  на измеримом пространстве $(E,\mathcal E)$, такая что $L(x,A)>0$ для всякого
  множества $A\in\mathcal E$ с $\varphi(A)>0$ и для всех $x\in E$. Цепь Маркова
  $\{X_n; n=0, 1, \ldots\}$ c $\varphi$-неприводимой переходной вероятностью
  $P(\cdot, \cdot)$ также называется $\varphi$-неприводимой.

  Стохастическое ядро $P(\cdot, \cdot)$ называется $\psi$-неприводимым, если оно
  $\varphi$-неприводимо для некоторой меры $\varphi(\cdot)$, а $\psi(\cdot)$
  есть максимальная определяющая неприводимость мера
\end{definition}

Среди всех начальных распределений цепи естественно выделить (если оно
существует) распределение $\pi(\cdot)$, такое что
\begin{equation}
\label{eq:inv}
\pi(B)=\int_E \pi(dx) P(x,D), \qquad B\in \mathcal E.
\end{equation}
Общая цепь Маркова с таким начальным распределением становится
\emph{стационарной} случайной последовательностью в строгом смысле: при каждом
фиксированном $m$ случайные векторы $(X_k, X_{k+1}, \ldots, X_{k+m})$ одинаково
распределены для всех $k=0$, $1$, \ldots. Такое распределение $\pi(\cdot)$
называют \emph{стационарным}. 

Если уравнению \eqref{eq:inv} удовлетворяет сигма-конечная мера $\pi(\cdot)$,
такая что $\pi(A)<\infty$ хотя бы для одного множества $A$ с $\psi(A)>0$, то
$\pi(\cdot)$ называется \emph{инвариантной}. 

\newpage
\section{Эргодичность цепи Маркова}

Продемонстрировать весь спектр вопросов, связанных с эргодичностью, можно на
примере \emph{конечной} цепи Маркова.

\begin{theorem}[эргодическая теорема \cite{Shyr}]
Пусть $Q=||p_{i,j}||$~--- матрица переходных вероятностей марковской цепи с
конечным множеством состояний $X=\{1, 2, \ldots, N\}$.\par
a) Если для некоторого $n_0$
\begin{equation}\label{sh:e21}
\min_{i,j} p^{(n_0)}_{i,j}>0,
\end{equation}
то найдутся числа $\pi_1$, \ldots, $\pi_N$ со свойством
\begin{equation}\label{sh:e22}
\pi_j>0, \qquad \sum_{j} \pi_j=1
\end{equation}
такие, что для каждого $j\in X$ и любого $i\in X$
\begin{equation}\label{sh:e23}
p^{(n)}_{i,j}\to \pi_j, \qquad n\to\infty.
\end{equation}

b) Обратно, если существуют числа $\pi_1$, \ldots, $\pi_n$, удовлетворяющие
уcловиям \eqref{sh:e22} и \eqref{sh:e23}, то найдется $n_0$ такое, что выполнено
условие  \eqref{sh:e21}.\par

c) Числа $(\pi_1, \ldots, \pi_N)$ из а) удовлетворяют системе уравнений
\[
\pi_j = \sum_{\alpha} \pi_\alpha p_{\alpha, j}, \qquad j=1, \ldots, N.
\]
\end{theorem}

Числа $\pi_1$, \ldots, $\pi_N$ являются, таким образом, инвариантным и
стационарным распределением вероятностей. Стационарное распределение с $\pi_j>0$
для всех $j=1$, \ldots, $N$ называется \emph{эргодическим}. Говорят, что цепь
Маркова обладает свойством \emph{эргодичности}, если пределы $\lim_{n}
p^{(n)}_{i,j}=\pi_j$ не только существуют, не зависят от $i$, образуют
распределение вероятностей и $\pi_j>0$ для всех $j=1$, \ldots, $N$.

В ходе доказательства эргодической теоремы для конечных цепей устанавливается,
что существует число $\rho$, $0<\rho<1$, такое что
\[
\left| p^{(n)}_{i,j}-\pi_j\right| \leq C \rho^n.
\]

Напомним определение расстояния полной вариации между распределениями
вероятностей $P_1(\cdot)$ и $\Pi_2(\cdot)$ на измеримом пространстве $(E,
\mathcal E)$:
\[
\sigma(P_1, P_2) = \sup_{A\in \mathcal E} |P_1(A)- P_2(A)|.
\]
Если меры $P_1$ и $P_2$ сосредоточены на одном и том же счетном множестве $x_1$,
$x_2$, \ldots{}, то 
\[
\sigma(P_1, P_2) = \frac{1}{2} \sum_{j} |P_1(\{ x_j\}) - P_2(\{x_j\})|.
\]
Используя это понятие можно сказать, что для эргодической цепи Маркова с
конечным числом состояний
\[
\sigma( P_n(x, \cdot), \pi(\cdot))\leqslant \tilde{C}\rho^n\to\infty,
\]
где $P_n(x,B)=\sum_{j\in B} p^{(n)}_{x,j}$, $\pi(B)=\sum_{j\in B} \pi_j$. 

Оказывается, что эргодическая цепь Маркова с конечным числом состояний
\emph{метрически транзититивна} (\cite{Dub}). Из метрической транзитивности
эргодической цепи Марковс с конечным числом состояний следует ее эргодичность,
как ее обычно вводят в теории стационарных в узком смысле процессов: для
произвольно функции $f(x)$ выполняется усиленный закон больших чисел
\[ 
\frac{f(X_0)+f(X_1)+\ldots+f(X_n)}{n+1} \to \sum_{j} f(j) \pi_j.
\]

\newpage

\begin{thebibliography}{99}
\bibitem{Shyr}
  Ширяев А.Н. Вероятность. В 2х кн. М.: МЦНМО, 2004.
\bibitem{Kemeni:Snell}
  Кемени Дж., Снелл Дж., Кнепп А. Счетные цепи Маркова. --- М.: Наука,
  Гл. ред. физ.-мат. лит.~--- 1987.~--- 416~с.
\bibitem{Hamilton} Hamilton J.D. Time Series Analysis. --- Princeton: Princeton
  University Press, 1994.~--- 814~p.
\bibitem{Dub}
  Дуб Дж.Л. Вероятностные процессы. --- М.: Изд-во иностр. лит., 1956. --- 606~с. 
\end{thebibliography}
\end{document}

\chapter*{Введение}							% Заголовок
\addcontentsline{toc}{chapter}{Введение}	% Добавляем его в оглавление

\newcommand{\actuality}{{\textbf\actualityTXT}}
\newcommand{\progress}{}
\newcommand{\aim}{{\textbf\aimTXT}}
\newcommand{\tasks}{\textbf{\tasksTXT}}
\newcommand{\novelty}{\textbf{\noveltyTXT}}
\newcommand{\influence}{\textbf{\influenceTXT}}
\newcommand{\methods}{\textbf{\methodsTXT}}
\newcommand{\defpositions}{\textbf{\defpositionsTXT}}
\newcommand{\reliability}{\textbf{\reliabilityTXT}}
\newcommand{\probation}{\textbf{\probationTXT}}
\newcommand{\contribution}{\textbf{\contributionTXT}}
\newcommand{\publications}{\textbf{\publicationsTXT}}

{\actuality} Теория массового обслуживания (теория очередей) занимается построением и анализом моделей для сложных систем, осуществляющих большое количество однотипных операций по обслуживанию различного рода требований (заявок, клиентов). Первые работы в этой области были мотивированы решением прикладных задач, связанных с организацией деятельности телефонных станций в начале XX века. Задачи, поставленные и рассмотренные Ф.В.~Йоханнсеном~\cite{Johannsen} и А.К.~Эрлангом~\cite{Erlang1909,Erlang1917}, заложили основу так называемой классической теории очередей. Дальнейшее развитие этой отрасли науки связано с именами таких ученых как 
Ф.~Поллачек, А.Н.~Колмогоров, А.Я.~Хинчин~\cite{Hinchin}, Б.В.~Гнеденко~\cite{GnedenkoKovalenko}, И.Н.~Коваленко, К.~Пальм~\cite{Palm}, Д.Дж.~Кендалл~\cite{Kendall}, Л.~Такач~\cite{Takacs}, Д.Р.~Кокс, У.Л.~Смит~\cite{KoksSmith}, Т.Л.~Саати~\cite{Saati}, Л.~Клейнрок~\cite{Kleinrok}, С.Н.~Бернштейн, Н.П.~Бусленко~\cite{Buslenko}, А.А.~Боровков~\cite{Borovkov}, В.С.~Королюк, Г.П.~Башарин~\cite{Basharin}, Г.П.~Климов~\cite{Klimov}, Ю.В.~Прохоров, А.Д.~Соловьев,  Б.А.~Севастьянов~\cite{Sevastitanov1969}, Н.Т.Дж.~Бейли~\cite{Bailey1952} и др. В их фундаментальных и прикладных работах закладываются основные понятия, формируются принципы и методы решения задач теории массового обслуживания. 

На протяжении всего XX века происходит бурный рост сферы услуг, а вместе с ней развивается и классическая теория. Важным стимулом такого развития служит огромное число задач, к которым применимы результаты теории массового обслуживания. Так, начиная с области телефонии, результаты теории очередей находят свое применение при исследовании систем управления наземным, водным и воздушным транспортом~\cite{Darroch1964,Helly1964,Haight1964,Gideon1999,Afanasieva2012}, систем организации медицинских учреждений~\cite{Bailey1954,Bailey1957,Flagle1962}, биологических систем~\cite{Kendall1952}, телекоммуникационных и компьютерных систем~\cite{Eisen1962,Vishnevsi2017}, процессов производства сложных объектов~\cite{Brigham1955}, в финансовой сфере~\cite{Albrecher2011} и т.~д.

Напомним, что одной из первых задач, поставленных перед теорией массового обслуживания, являлась задача определения минимального числа телефонных каналов, обеспечивающих удовлетворительное обслуживание телефонных абонентов. Подобные оптимизационные цели существуют у большинства прикладных исследований в рамках теории очередей. Так, в некоторой идеализации система массового обслуживания есть система, которая, находясь под действием различных случайных, неопределенных, контролируемых факторов, осуществляет операции по обслуживанию требований. При этом возможно определить различные критерии эффективности осуществления операции: среднее время пребывания требования в системе, вероятность простоя обслуживающих приборов, вероятность отказа, среднее количество занятых приборов, средняя длина очереди, коэффициент загрузки системы, производительность системы и т.~д. Целью исследования таких систем является определение способов достижения наибольшей прибыли. В зависимости от выбранного критерия эффективности по-разному понимается и наибольшая прибыль, наибольшая эффективность работы системы. В связи с такой постановкой задачи теория массового обслуживания неразрывно связана с отраслью исследования операций. Такая связь наблюдается, например, в отечественных и зарубежных работах~\cite{Buslenko1968,Bailey1957,Flagle1962}. Заметим, что задачи теории очередей в этом случае понимаются как задачи организационно-управленческого  характера, направленные на наиболее оптимальное использование имеющихся ресурсов. 

Следует также обратить внимание на связь теории очередей и математической кибернетики. В работе~\cite{Lyapunov} авторы выделяют понятие управляющей системы как одно из ключевых понятий математической кибернетики. Кибернетика представляется как «наука об общих закономерностях строения управляющих систем и течения процессов управления». При изучении конкретных управляющих систем кибернетика взаимодействует со многими другими областями знаний. Как отмечено в работе~\cite{Lyapunov}, к таким областям относится и теория массового обслуживания, поскольку система обслуживания может рассматриваться как управляющая система. В связи с этим привлечение аппарата и результатов математической кибернетики, исследования операций и других дисциплин представляется актуальным и позволяет получать новые результаты.

Оптимизационная постановка задачи всегда требует от исследователя возможности управления системой. В связи с этим, начиная со второй половины XX века, появляются работы, посвященные теории управляемых систем обслуживания. Понятие управляемой системы массового обслуживания было введено О.И. Бронштейном и В.В. Рыковым в работе~\cite{BronshteinRykov}. Так, было отмечено, что в управляемых системах массового обслуживания можно выделить элементы, допускающие применение управляющих воздействий.  Каждый подобный элемент характеризуется набором параметров. В свою очередь, выбор значений управляющих параметров является стратегией управления. Кроме того, изучению управляемых систем обслуживания посвящены работы Н.М. Воробьева~\cite{Vorobiev1967}, Б.Г. Питтеля~\cite{Pittel1972}, А.Ф. Терпугова, В.В. Рыкова~\cite{Rykov1975} и др.

В классическом варианте система массового обслуживания содержит четыре обязательных элемента: входной поток, дисциплина образования очереди, закон обслуживания заявок, структура обслуживающего устройства. Каждый из этих основных элементов может подвергаться управляющим воздействиям. Кроме того, каждый из элементов может обладать переменной структурой, что выводит исследователя из рамок классической теории очередей. В связи с этим, отметим здесь несколько направлений исследований, связанных с тематикой диссертационной работы.

Одним из важных направлений является изучение входных потоков системы. В первых работах по теории массового обслуживания самой распространенной моделью входного потока служил простейший поток, или поток Пуассона. Действительно, многие реальные потоки требований обладают свойствами стационарности, ординарности и отсутствия последействия. Важным моментом в понимании распространенности модели пуассоновского потока являются предельные теоремы Пуассона для серий независимых испытаний с малой вероятностью успеха. Напомним, что в таком  случае пуассоновское распределение возникает в качестве предельного для биномиального распределения при увеличении количества независимых испытаний~\cite{Kingman}. Так, реальные потоки требований формируются, как правило, из большого числа независимых требований. Подобные рассуждения применимы, например, к транспортным потокам на крупных магистралях, потокам клиентов в крупных супермаркетах, потокам пациентов в поликлинику в период отсутствия массовых заболеваний, потокам отказов элементов сложных технических устройств и другим потокам различной физической природы. В работах~\cite{Heit,Breiman1963} изучается пространственная и временная характеристики транспортного потока и дается обоснование пуассоновской модели для такого потока. Однако довольно часто реальные потоки составлены из неоднородных, зависимых требований. Так, например, в работах~\cite{FedotkinNonlocal,Fedotkin2012} предлагается строить нелокальное описание для таких потоков. Действительно, часто влияние внешних условий на формирование потока требований приводит к тому, что проявляется неоднородность требований. При исследовании различных механизмов образования потока заявок возникают такие модели, как поток Гнеденко--Коваленко~\cite{Fedotkin2013}, поток Бартлетта~\cite{Fedotkin1996}. Актуальность диссертационного исследования в этом направлении обусловлена тем, что в работе предлагается механизм образования пачки требований ограниченной длины и на его основе строится модель реальных потоков в виде неординарных пуассоновских потоков. Исследованию систем массового обслуживания с неординарными потоками, или потоками пачек, посвящены также работы~\cite{Foster1961,Pechinkin1990,Monsik2009,Monsik2010}. 

Часто входные потоки системы являются управляемым элементом системы массового обслуживания. Так, например, можно отметить ряд работ, посвященных системам с управляемым входным потокам~\cite{Natan1966,Kovalenko1968,Kovalenko1971}. В последние десятилетия особую роль играют исследования потоков, управляемых марковскими процессами: MAP-потоки (Markovian Arrival Process) и их различные вариации. В общем случае процесс поступления заявок контролируется процессом, который можно моделировать   с помощью процесса Маркова. К таким управляющим процессам часто относят влияние внешней среды. Так, при смене состояния внешней среды может существенно поменяться структура или интенсивность потока. Впервые понятие марковского потока было предложено М.~Ньютсом~\cite{Neuts1979} и уточнено Д.~Лукантони~\cite{Lucantoni1990,Lucantoni1993}. Дальнейшее распространение марковские модели получили благодаря исследованиям зарубежных и отечественных авторов~\cite{Asmussen1993,Bocharov1997,Abolnikov2007,Nazarov2010,Moiseev2013,Dudin2015}. Влияние внешней случайной среды на процесс формирования потока требований приводит также к дважды стохастическим моделям~\cite{Grandell1976}, при которых параметры входного потока меняются со временем. Такие потоки изучались, например, в работах~\cite{Gorcev2013,Zorin2005,Golovko2009}. В литературе рассмотрены и другие модели входных потоков: поток авторегрессионного типа~\cite{Leontiev2016}, эрланговский	поток~\cite{Ushakov1977}, нестационарный поток~\cite{Davis1995} и т.~д.  

Следующее направление исследований связано с системами массового обслуживания с переменной структурой обслуживающего устройства. Пионерские работы в этой области принадлежат М.А. Федоткину и М.Г. Теплицкому~\cite{Teplicki1968,Teplicki1969}. В работах~\cite{Neimark1966,Fedotkin1969} методами теории массового облуживания и математической кибернетики решалась задача управления потоками машин на перекрестке. В качестве обслуживающего устройства рассматривался перекресток с установленным автоматом-светофором. При этом светофор может находиться в одном из множества состояний, меняющихся согласно некоторому закону. Каждое состояние автомата характеризуется определенным режимом обслуживания входных потоков. Были найдены оптимальные значения для управляющих параметров светофора. Далее были рассмотрены системы с различными законами смены состояний обслуживающего устройства: например, в работах~\cite{Proidakova2008,Fedotkin2014,Zorin2014} изучался циклический алгоритм для различных входных потоков, в работе~\cite{Zorin2017}~-- алгоритм с петлей, в~\cite{Kuvykina1990} -- алгоритм с упреждением, в работах~\cite{Kudelin1996,Litvak2000,Golysheva2010} -- различные иные адаптивные алгоритмы. Заметим, что подобные модели дают адекватное математическое описание многих реальных сложных управляемых процессов обслуживания, учитывающих воздействие случайных факторов. 

Здесь же следует отметить ряд работ, изучающих алгоритмическое управление потоками в рамках исследования операций. Наиболее наглядным приложением таких моделей являются по-прежнему системы управления дорожным транспортом. Например, в работе~\cite{Dunne1964} рассматривается линейный управляющий алгоритм, определяются условия стабильности управления потоками, изучаются условия, при которых минимизируются средние задержки в обслуживании. В работе~\cite{Gordon1969} изучается адаптивный алгоритм, использующий информацию о длинах очередей. При построении модели используется z-преобразование, определяются условия стабильности. В работе~\cite{Day2012} рассматривается совместное использование двух управляющих алгоритмов (алгоритм с обратной связью и алгоритм с упреждением), обосновывается эффективность такого подхода.  Различные адаптивные алгоритмы и их комбинации в применении к управлению потоками данных рассматриваются в работах~\cite{Vasilakos1990,Cotton1995,Mason1999,Kokkonis2016}. Конечной прикладной целью исследований является определение оптимальной стратегии управления системой. Для этого в~\cite{Mason1999}, например, решается нелинейная оптимизационная задача с применением условий Куна-Таккера. Работа~\cite{Huang2015} содержит исследование системы обслуживания потоков разноклассовых клиентов в отделении неотложной помощи. Управление потоками в данной работе осуществляется на основе обратной связи по количеству заявок в очереди ожидания и времени пребывания в системе заявок, находящихся на обслуживании. Устанавливаются  условия асимптотической оптимальности с использованием методов компьютерной имитации. Отметим, что многие подобные исследования существенно опираются на физические характеристики и особенности системы. Если постановка задачи формулируется при изучении реальной физической системы, то зачастую результаты исследований сложно перенести на задачи другой физической природы. В этом смысле диссертационная работа является актуальной, поскольку предлагает рассматривать управляющие системы и алгоритмы, абстрагируясь от физической постановки задачи.


Следующее направление связано с исследованием предельного поведения систем массового обслуживания и условий ее стационарности. Здесь следует отметить работы таких авторов, как А.А. Боровков~\cite{Borovrkov1964,Borovrkov1980}, Л.Г. Афанасьева~\cite{Afanasieva2008,Afanasieva2011}, Е.В. Булинская, В. Уитт~\cite{Whitt1971,Whitt1982}, Дж. Дэвис~\cite{Davis1995} и др. Многие из подобных исследований направлены на асимптотический анализ операционных характеристик системы (время ожидания, размер очереди, число требований в системе и т.~п.). Кроме того, как теоретический, так и прикладной интерес представляют работы, в которых определяются условия существования стационарного режима в системах обслуживания. В качестве примера таких работ можно привести исследования~\cite{Loynes1962,Davis1972,Choudhury1995,Whitt2014}. Частой методологией отыскания условий являются интегральные преобразования функций, характеризующих систему. Также используются известные результаты теории управления и теории цепей Маркова, например, критерий устойчивости Найквиста-Михайлова~\cite{Davis1972}, теорема эргодичности Мустафы~\cite{Nazarov2011} и др. Нередко результатом подобных исследований являются условия существования стационарного режима, которые сложно проверить для реальных систем. Отметим, что свойство стационарности системы сопряжено с понятием ее управляемости. Исследование стационарного режима является важным этапом при решении задачи оптимального управления системой. В связи с этим желательным является получение условий существования стационарного режима, зависящих от управляющих параметров. В этом отношении следует обратить внимание на итеративно-мажорантный метод, разработанный М.А. Федоткиным~\cite{Fedotkin1988,Fedotkin1989}. Данный метод позволяет получать легко проверяемые условия существования стационарного режима для управляющих систем конфликтного обслуживания. Например, такая методология применялась в~\cite{Proidakova2007} при исследовании системы с приоритетным направлением, \cite{Zorin2008} -- при изучении системы, находящейся под влиянием внешней среды, \cite{LitvakDissertation} -- при исследовании системы, использующей информацию о количестве заявок, интервалах поступления и очередности подхода заявок и др. В диссертационной работе указанный метод применяется для отыскания условий существования стационарного режима для двух новых систем управления конфликтными потоками.

Также следует отметить направление исследований, связанное с приоритетными системами. Системы с абсолютным, относительным или иным приоритетом в разное время изучались И.М. Духовным~\cite{Duhovny1969}, О.И. Бронштейном~\cite{Bronshtein1971}, А.В. Печинкиным~\cite{Pechinkin1990}, П.П. Бочаровым~\cite{Bocharov1997}, В.Г. Ушаковым~\cite{Ushakov1977,Leontiev2016}, а также в работах~\cite{Volkovinski19891,Volkovinski19892,Mishkoi2008,Ushakov2012} и др. Системы, в которых входящие требования разнородны и могут быть разделены на классы, получают широкое распространение. В частности, это объясняется тем, что приоритетные системы служат математическими моделями для информационно-вычислительных систем~\cite{Bronshtein1976} и современных мультисерверных коммуникационных~\cite{Vishnevski2005} и компьютерных~\cite{Vishnevski2003} сетей. В диссертационной работе рассматривается как система с однородными входными потоками, так и система, в которой потоки различаются по своему приоритету. Во втором случае для управления потоками необходим адаптивный алгоритм. Кроме того, характеристика эффективности работы системы при решении оптимизационной задачи также учитывает различный приоритет заявок.

Как было указано выше, частой целью изучения систем обслуживания, выполняющих также операции по управлению, является поиск оптимальной стратегии управления. Так, оптимизационные задачи ставятся и решаются, например, в работах~\cite{Klimov1966,Kovalenko1968,Kredepcer1970,Lanin1972,Vasilakos1990,Mason1999,Cruz2014,Babicheva2015,Huang2015}. Постановка задачи в первых классических работах по теории очередей сводилась к поиску оптимального числа обслуживающих приборов, минимизирующего среднее время ожидания клиентов. Решение при этом находилось аналитически. Со временем системы, моделируемые и изучаемые методами теории массового обслуживания и исследования операций, значительно усложнялись. В связи с этим возникает необходимость в новых критериях оценки качества функционирования системы, а также в новых методах исследования. Обзор некоторых современных методов оценки производительности системы приведен, например, в~\cite{Cruz2014}. Одним из самых распространенных методов при решении подобных оптимизационных задач на текущий момент является метод имитационного моделирования~\cite{Simulation}. Компьютерные имитационные модели позволяют учитывать достаточно большое число факторов, которые с трудом могут быть учтены при аналитическом исследовании в силу его возрастающей сложности. Кроме того, преимущество имитационных моделей связано с возможностью исследовать различные сценарии работы управляемых систем, сравнительно легко адаптировать модели к изменениям в физической постановке задачи. С развитием параллельного программирования и увеличением вычислительных способностей компьютеров имеется возможность получать достоверные оценки качества и решать оптимизационные задачи в более короткие сроки. В диссертационной работе аналитические методы применяются наряду с численным исследованием путем имитационного моделирования. Такое объединение методологий представляется актуальным и увеличивает достоверность результатов.

{\aim} Целями данной работы являются: 1)~построение и исследование математической модели тандема управляющих систем обслуживания по циклическому алгоритму с продлением; 2)~построение, реализация и анализ имитационной модели систем, осуществляющих циклическое управление с продлением тандемом перекрестков.

Для~достижения поставленных целей решаются следующие задачи:

1. Построение строгой вероятностной модели тандема управляющих систем с помощью явного построения вероятностного пространства и поточечного задания необходимых для исследования случайных величин и элементов.

2. Анализ построенной вероятностной модели, получение условий существования стационарного режима в различных подсистемах тандема.

3. Разработка имитационной модели тандема, определение момента достижения системы квазистационарного режима, анализ зависимости условий стационарности от управляющих параметров.



{\novelty} Основные результаты являются новыми и состоят в следующем:

1. Построена строгая вероятностная модель тандема управляющих систем с немгновенным перемещением требований между ними, управление в которых осуществляется по циклическому алгоритму и алгоритму с продлением. 

2. Изучены эргодические свойства построенной модели, найдены условия существования стационарного режима для очередей первичных требований, а также для промежуточной очереди.

3. Разработана и реализована имитационная модель для тандема

4. Проведено исследование вероятностной и имитационной моделей, и определена расширенная область стационарности системы при алгоритме с продлением.




{\influence} Научная значимость работы заключается в построении строгой вероятностной модели 
для качественно нового вида управляющей системы и в последовательном исследовании ее эргодических свойств. Успешно примененный в работе метод нелокального описания процессов существенно расширяет множество поддающихся исследованию реальных систем массового обслуживания. Строгая математическая модель позволяет оперировать существующим, хорошо разработанным вероятностным аппаратом для нахождения условий стационарности и нахождения оптимального управления системой. 
 Разработанные модели дают базу для изучения более комплексных тандемных систем, систем с более сложными входными потоками и алгоритмами управления.

Практическая значимость исследования состоит в том, что изученная управляющая система является адекватным описанием реальной системы тандема перекрестков, а также других сетей, состоящих из двух узлов с перемещающимися между ними требованиями и циклическими алгоритмами обслуживания с продлением на узлах.




{\methods} Методология диссертационной работы базируется на представлении стохастических систем массового обслуживания в виде кибернетических управляющих систем. Применение принципов кибернетического подхода позволяет выделить в изучаемых системах ключевые блоки, структурировать информацию о законах функционирования блоков и основных связях между ними. Для описания входных потоков был примен метод нелокального описания, что сделало возможным более глубокое математическое изучение рассматриваемых объектов. В работе используется аппарат теории вероятностей, теории массового обслуживания, исследования операций, теории управляемых марковских процессов, теории функций комплексного переменного. Также применяются методы математической статистики, матричных вычислений и теории систем линейных алгебраических уравнений. При разработке имитационных моделей использовался язык программирования C++. Для визуализации результатов некоторых численных исследований использовался язык Python.


{\defpositions}

1. Методика построения вероятностного пространства для тандема систем с немгновенным перемещением требований между ними.

2. Методика нахождения условий существования стационарного режима в системах управления потоками неоднородных требований с циклическим алгоритмом и алгоритмом с продлением.

3. Метод определения момента достижения управляемой системой обслуживания квазистационарного режима.





{\probation} Достоверность полученных результатов обеспечивается строгим применением используемого математического аппарата, проведением статистических и численных исследований. Результаты работы находятся в соответствии с результатами, полученными ранее другими авторами при исследовании управляющих систем обслуживания.

Основные результаты диссертации докладывались и обсуждались на следующих  конференциях.
\begin{enumerate}
    \item Международная научная конференция <<Теория вероятностей, случайные процессы, математическая статистика и приложения>> (Минск, Республика Беларусь, 2015 г.).
    \item IX Международная конференция <<Дискретные модели в теории управляющих систем>> (Москва и Подмосковье, 2015 г.).
\item 8-я международная научная конференция <<Распределенные компьютерные и коммуникационные сети: управление, вычисление, связь>> DCCN-2015 (Москва, 2015 г.).
\item Международная научная конференция <<Distributed Computer and Communication Networks>> DCCN 2016 (Москва, 2016 г.).
\item XVIII Международная конференция <<Проблемы теоретической кибернетики>> (Пенза, 2017 г.).
\item XVI Международная конференция имени А.Ф. Терпугова <<Информационные технологии и математическое моделирование>> ИТММ-2017 (Казань, 2017 г.).
\item  20-я международная научная конференция <<Распределенные компьютерные и телекоммуникационные сети: управление, вычисление, связь>> DCCN-2017 (Москва, 2017 г.).
\item IX Московская международная конференция по исследованию операций (Москва, 2018~г.).
\end{enumerate}


{\contribution} В совместных публикациях научному руководителю принадлежит постановка задачи и общее редактирование работ. Все исследования выполнены автором диссертации лично, все полученные результаты принадлежат автору. 


\ifnumequal{\value{bibliosel}}{0}{% Встроенная реализация с загрузкой файла через движок bibtex8
    \publications\ Основные результаты по теме диссертации изложены в XX печатных изданиях, 
    X из которых изданы в журналах, рекомендованных ВАК, 
    X "--- в тезисах докладов.%
}{% Реализация пакетом biblatex через движок biber
%Сделана отдельная секция, чтобы не отображались в списке цитированных материалов
    %\begin{refsection}%
        %\printbibliography[heading=countauthornotvak, env=countauthornotvak, keyword=biblioauthornotvak, section=1]%
        %\printbibliography[heading=countauthorvak, env=countauthorvak, keyword=biblioauthorvak, section=1]%
        %\printbibliography[heading=countauthorconf, env=countauthorconf, keyword=biblioauthorconf, section=1]%
        %\printbibliography[heading=countauthor, env=countauthor, keyword=biblioauthor, section=1]%
        %\publications\ Основные результаты по теме диссертации изложены в \arabic{citeauthor} печатных изданиях\nocite{bib1,bib2}, 
        %\arabic{citeauthorvak} из которых изданы в журналах, рекомендованных ВАК\cite{vestnikUNN,vestnikVGAVT1,vestnikVGAVT2,vestnikTGU}, 
        %\nocite{DCCN2010,Minsk2011,Novgorod2011,Novosibirsk2011,DCCN2013,Kazan,Minsk2015,RachinskayaStatistics,Dm2015,DCCN2015,Dm2016,Penza2017,DCCN2017,Tomsk2017,Soloviev2017}\arabic{citeauthorconf} "--- в тезисах докладов  \cite{DCCN2010,Minsk2011,Novgorod2011,Novosibirsk2011,DCCN2013,Kazan,Minsk2015,RachinskayaStatistics,Dm2015,DCCN2015,Dm2016,Penza2017,DCCN2017,Tomsk2017,Soloviev2017}.
        	\publications\ Основные результаты по теме диссертации изложены в 10 работах, 
	1 из них "--- в журнале, рекомендованном ВАК для защиты по специальности 01.01.09,
	2 "--- в библиографической базе Scopus, 2 "--- в библиографической базе Web of Science, 1 "--- в журнале, рекомендованном ВАК для защиты по смежной специальности 05.13.01, 9 "--- в библиографической базе РИНЦ,
	8 "--- в тезисах докладов. 
	%  \end{refsection}

}



 % Характеристика работы по структуре во введении и в автореферате не отличается (ГОСТ Р 7.0.11, пункты 5.3.1 и 9.2.1), потому её загружаем из одного и того же внешнего файла, предварительно задав форму выделения некоторым параметрам

\textbf{Объем и структура работы.} Диссертация состоит из~введения, четырёх глав, заключения и~двенадцати приложений.
%% на случай ошибок оставляю исходный кусок на месте, закомментированным
%Полный объём диссертации составляет  \ref*{TotPages}~страницу с~\totalfigures{}~рисунками и~\totaltables{}~таблицами. Список литературы содержит \total{citenum}~наименований.
%
Полный объём диссертации составляет
\formbytotal{TotPages}{страниц}{у}{ы}{}, включая
\formbytotal{totalcount@figure}{рисун}{ок}{ка}{ков}. % и
%\formbytotal{totalcount@table}{таблиц}{у}{ы}{}.   
 Список литературы содержит  
\formbytotal{citenum}{наименован}{ие}{ия}{ий}.

Во \textit{введении} приводится обзор научной литературы по изучаемой проблематике, обосновывается актуальность проводимых исследований и дается краткая характеристика работы, содержащая цели, задачи и основные результаты исследований, сведения об апробации результатов, а также теоретическую и практическую значимость работы.

\textit{Первая глава} посвящена построению и изучению математической модели потока неоднородных требований. Часто под влиянием внешних факторов между требованиями потока образуется зависимость, что приводит к скоплению требований в группы (пачки). На примере транспортного потока в разделе~\ref{1:sect1} представлен механизм образования группы требований. Получена система дифференциальных уравнений, определяющая динамику изменения количества требований в пачке с течением времени. В разделе~\ref{1:sect2} найдено распределение для числа требований в пачке в установившемся режиме в случае, когда пачка может состоять не более, чем из трех требований. В разделе~\ref{1:sect3} произведен переход от модели отдельной пачки к модели потока пачек, описываемого с помощью неординарного пуассоновского потока. Найдены одномерные распределения процесса $\{\eta(t)\colon t\geq 0\}$, где величина $\eta(t)$ считает количество требований, поступивших по потоку за промежуток $[0, t)$. Определены выражения для числовых характеристик величины $\eta(t)$, изучены их экстремальные значения. В разделе~\ref{1:sect4} предложены процедуры и методы анализа данных реальных потоков для обоснования корректности аппроксимации потоков неоднородных требований с помощью неординарных пуассоновских потоков.

Раздел~\ref{2:sect1} \textit{второй главы} содержит описание класса кибернетических систем, осуществляющих операции по обслуживанию требований и управлению конфликтными потоками. Системы внутри класса различаются двумя составляющими: видом входных потоков и управляющим алгоритмом. Далее вся вторая глава посвящена исследованию системы с входными потоками неоднородных требований и циклическим управляющим алгоритмом. В разделе~\ref{2:sect2} построена математическая модель такой системы в виде управляемой многомерной цепи Маркова. Рекуррентные зависимости для одномерных распределений указанного марковского процесса получены в разделе~\ref{2:sect3}. Произведена классификация состояний цепи Маркова. Основным результатом главы является критерий существования в системе стационарного режима в виде соотношений для параметров системы.

Аналогичные исследования проведены в \textit{третьей главе} для системы адаптивного управления потоками с различными приоритетами и интенсивностями. Рассмотрен алгоритм с пороговым приоритетом и возможностью продления обслуживания. Алгоритм обеспечивает обратную связь по количеству требований высокоприоритетного потока в очереди ожидания начала обслуживания. Описание потоков и управляющего алгоритма приведено в разделе~\ref{3:sect1}. Математическая модель системы в виде многомерной однородной цепи Маркова построена в разделе~\ref{3:sect2} и изучена в разделе~\ref{3:sect3}. Произведен анализ пространства состояний цепи Маркова, доказано утверждение о существовании предельных распределений. В завершении главы найдены условия существования стационарного режима в системе по отдельным потокам.

В \textit{четвертой главе} производится численное исследование управляющих систем обслуживания путем имитационного моделирования. Целью исследования является нахождение квазиоптимальной стратегии управления, т.~е. таких значений управляющих параметров системы, при которых достигается минимальное значение оценки среднего времени ожидания начала обслуживания произвольным требованием. В разделе~\ref{4:sect2} приведено описание построенных имитационных моделей. В разделе~\ref{4:sect3} предложен метод определения момента достижения системой квазистационарного режима. Также в четвертой главе дано описание алгоритмов определения квазиоптимальных стратегий управления для  систем, рассмотренных в главах~\ref{chapt2} и~\ref{chapt3}.

В \textit{заключении} приведены основные результаты работы и указаны возможные направления развития исследований.

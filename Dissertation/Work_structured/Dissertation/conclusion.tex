\chapter*{Заключение}						% Заголовок
\addcontentsline{toc}{chapter}{Заключение}	% Добавляем его в оглавление

%% Согласно ГОСТ Р 7.0.11-2011:
%% 5.3.3 В заключении диссертации излагают итоги выполненного исследования, рекомендации, перспективы дальнейшей разработки темы.
%% 9.2.3 В заключении автореферата диссертации излагают итоги данного исследования, рекомендации и перспективы дальнейшей разработки темы.
%% Поэтому имеет смысл сделать эту часть общей и загрузить из одного файла в автореферат и в диссертацию:

В приведенной работе был рассмотрен тандем управляющих систем, управление в которых осуществляется по циклическому алгоритму и алгоритму с продлением. Основные результаты работы заключаются в следующем.

    \begin{enumerate}
        \item Построена строгая математическая модель тандема с циклическим алгоритмом управления и алгоритмом с продлением. Отличительной особенностью системы также является немгновенность перемещения требований между системами. 
        \item Доказана марковость случайной последовательности, включающей длину низкоприоритетной очереди. Проведена классификация состояний цепи по арифметическим свойствам переходных вероятностей этой последовательности. А также найдены достаточное и необходимое условия существования стационарного распределения.
        \item Проведен аналогичный анализ для случайной последовательности, включающей очереди первичных требований: доказана ее марковость, проведена классификация состояний и найдено достаточное условие существования стационарного распределения.
        \item Найдено условие ограниченности для последовательности математических ожиданий $    \{( E\varkappa_{4,i}); i \geqslant 0\}$.
        \item Разработана имитационная модель для изучения исходной системы и написана программа ее реализующая.
        \item На основе имитационной модели были подтверждены и расширены результаты, полученные теоретически.
    \end{enumerate}



{\aim} Целями данной работы являются: 1)~построение и исследование математической модели тандема управляющих систем обслуживания по циклическому алгоритму с продлением; 2)~построение, реализация и анализ имитационной модели систем, осуществляющих циклическое управление с продлением тандемом перекрестков.

Для~достижения поставленных целей решаются следующие задачи:

1. Построение строгой вероятностной модели тандема управляющих систем с помощью явного построения вероятностного пространства и поточечного задания необходимых для исследования случайных величин и элементов.

2. Анализ построенной вероятностной модели, получение условий существования стационарного режима в различных подсистемах тандема.

3. Разработка имитационной модели тандема, определение момента достижения системы квазистационарного режима, анализ зависимости условий стационарности от управляющих параметров.



{\novelty} Основные результаты являются новыми и состоят в следующем:

1. Построена строгая вероятностная модель тандема управляющих систем с немгновенным перемещением требований между ними, управление в которых осуществляется по циклическому алгоритму и алгоритму с продлением. 

2. Изучены эргодические свойства построенной модели, найдены условия существования стационарного режима для очередей первичных требований, а также для промежуточной очереди.

3. Разработана и реализована имитационная модель для тандема

4. Проведено исследование вероятностной и имитационной моделей, и определена расширенная область стационарности системы при алгоритме с продлением.




{\influence} Научная значимость работы заключается в построении строгой вероятностной модели 
для качественно нового вида управляющей системы и в последовательном исследовании ее эргодических свойств. Успешно примененный в работе метод нелокального описания процессов существенно расширяет множество поддающихся исследованию реальных систем массового обслуживания. Строгая математическая модель позволяет оперировать существующим, хорошо разработанным вероятностным аппаратом для нахождения условий стационарности и нахождения оптимального управления системой. 
 Разработанные модели дают базу для изучения более комплексных тандемных систем, систем с более сложными входными потоками и алгоритмами управления.

Практическая значимость исследования состоит в том, что изученная управляющая система является адекватным описанием реальной системы тандема перекрестков, а также других сетей, состоящих из двух узлов с перемещающимися между ними требованиями и циклическими алгоритмами обслуживания с продлением на узлах.




{\methods} Методология диссертационной работы базируется на представлении стохастических систем массового обслуживания в виде кибернетических управляющих систем. Применение принципов кибернетического подхода позволяет выделить в изучаемых системах ключевые блоки, структурировать информацию о законах функционирования блоков и основных связях между ними. Для описания входных потоков был примен метод нелокального описания, что сделало возможным более глубокое математическое изучение рассматриваемых объектов. В работе используется аппарат теории вероятностей, теории массового обслуживания, исследования операций, теории управляемых марковских процессов, теории функций комплексного переменного. Также применяются методы математической статистики, матричных вычислений и теории систем линейных алгебраических уравнений. При разработке имитационных моделей использовался язык программирования C++. Для визуализации результатов некоторых численных исследований использовался язык Python.


{\defpositions}

1. Методика построения вероятностного пространства для тандема систем с немгновенным перемещением требований между ними.

2. Методика нахождения условий существования стационарного режима в системах управления потоками неоднородных требований с циклическим алгоритмом и алгоритмом с продлением.

3. Метод определения момента достижения управляемой системой обслуживания квазистационарного режима.





{\probation} Достоверность полученных результатов обеспечивается строгим применением используемого математического аппарата, проведением статистических и численных исследований. Результаты работы находятся в соответствии с результатами, полученными ранее другими авторами при исследовании управляющих систем обслуживания.

Основные результаты диссертации докладывались и обсуждались на следующих  конференциях.
\begin{enumerate}
    \item Международная научная конференция <<Теория вероятностей, случайные процессы, математическая статистика и приложения>> (Минск, Республика Беларусь, 2015 г.).
    \item IX Международная конференция <<Дискретные модели в теории управляющих систем>> (Москва и Подмосковье, 2015 г.).
\item 8-я международная научная конференция <<Распределенные компьютерные и коммуникационные сети: управление, вычисление, связь>> DCCN-2015 (Москва, 2015 г.).
\item Международная научная конференция <<Distributed Computer and Communication Networks>> DCCN 2016 (Москва, 2016 г.).
\item XVIII Международная конференция <<Проблемы теоретической кибернетики>> (Пенза, 2017 г.).
\item XVI Международная конференция имени А.Ф. Терпугова <<Информационные технологии и математическое моделирование>> ИТММ-2017 (Казань, 2017 г.).
\item  20-я международная научная конференция <<Распределенные компьютерные и телекоммуникационные сети: управление, вычисление, связь>> DCCN-2017 (Москва, 2017 г.).
\item IX Московская международная конференция по исследованию операций (Москва, 2018~г.).
\end{enumerate}


{\contribution} В совместных публикациях научному руководителю принадлежит постановка задачи и общее редактирование работ. Все исследования выполнены автором диссертации лично, все полученные результаты принадлежат автору. 


\ifnumequal{\value{bibliosel}}{0}{% Встроенная реализация с загрузкой файла через движок bibtex8
    \publications\ Основные результаты по теме диссертации изложены в XX печатных изданиях, 
    X из которых изданы в журналах, рекомендованных ВАК, 
    X "--- в тезисах докладов.%
}{% Реализация пакетом biblatex через движок biber
%Сделана отдельная секция, чтобы не отображались в списке цитированных материалов
    %\begin{refsection}%
        %\printbibliography[heading=countauthornotvak, env=countauthornotvak, keyword=biblioauthornotvak, section=1]%
        %\printbibliography[heading=countauthorvak, env=countauthorvak, keyword=biblioauthorvak, section=1]%
        %\printbibliography[heading=countauthorconf, env=countauthorconf, keyword=biblioauthorconf, section=1]%
        %\printbibliography[heading=countauthor, env=countauthor, keyword=biblioauthor, section=1]%
        %\publications\ Основные результаты по теме диссертации изложены в \arabic{citeauthor} печатных изданиях\nocite{bib1,bib2}, 
        %\arabic{citeauthorvak} из которых изданы в журналах, рекомендованных ВАК\cite{vestnikUNN,vestnikVGAVT1,vestnikVGAVT2,vestnikTGU}, 
        %\nocite{DCCN2010,Minsk2011,Novgorod2011,Novosibirsk2011,DCCN2013,Kazan,Minsk2015,RachinskayaStatistics,Dm2015,DCCN2015,Dm2016,Penza2017,DCCN2017,Tomsk2017,Soloviev2017}\arabic{citeauthorconf} "--- в тезисах докладов  \cite{DCCN2010,Minsk2011,Novgorod2011,Novosibirsk2011,DCCN2013,Kazan,Minsk2015,RachinskayaStatistics,Dm2015,DCCN2015,Dm2016,Penza2017,DCCN2017,Tomsk2017,Soloviev2017}.
        	\publications\ Основные результаты по теме диссертации изложены в 10 работах, 
	1 из них "--- в журнале, рекомендованном ВАК для защиты по специальности 01.01.09,
	2 "--- в библиографической базе Scopus, 2 "--- в библиографической базе Web of Science, 1 "--- в журнале, рекомендованном ВАК для защиты по смежной специальности 05.13.01, 9 "--- в библиографической базе РИНЦ,
	8 "--- в тезисах докладов. 
	%  \end{refsection}

}



